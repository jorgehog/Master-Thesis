\chapter{Quantum Monte Carlo}

\section{Modeling Diffusion}

Like any phenomena involving a density function, or distribution, Quantum Mechanics can be modeled by diffusion. In Quantum Mechanics, the distribution is given by $|\p|^2$, the Wave function squared. The diffusing elements of interest are the particles making up our system. The idea is to have an ensemble of \textit{Random Walkers} in which each walker represents a position is space (and time for time-dependent studies). Averaging values over the paths of the ensemble will yield average values corresponding to the probability distribution governing the movement of individual walkers. 

Such random movement is referred to as a \textit{Brownian motion}, named after the British Botanist R. Brown, originating from his experiments on plant pollen dispersed in water. \textit{Markow chains} are a subtype of Brownian motion, where a walkers next move is independent of previous moves. This is the stochastic process in which Quantum Monte Carlo is described.

The purpose of this section is to motivate the use of diffusion theory in Quantum Mechanics, and to derive the sampling rules needed in order to model Quantum Mechanical distributions by diffusion of random walkers correctly. I will be using natural units, that is $\hbar$, $m_e$, etc. are all set to unity, in order to simplify the expressions.

\subsection{The Diffusion Equation}

\subsubsection{The simple case}

Eq.~(\ref{Eq:diffusionSimple}) is the simplest form of a diffusion equation, that is, the case with a linear \textit{diffusion constant}, $D$, and no drift terms.

\begin{equation}
 \frac{\partial P(\vec r, t)}{\partial t} = D\nabla^2 P(\vec r, t) 
 \label{Eq:diffusionSimple}
\end{equation}

From experiment there is a strong indication that the \textit{probability flux}, $j(\vec r, t)$, is proportional to the gradient of the distribution \cite{morten}, that is

\begin{equation}
 j(\vec r, t) = -D\nabla|\p|^2.
 \label{eq:experimentalJ}
\end{equation}

If we on the other hand look at the time derivative of the Wave function squared, and insert expressions for the Schr\"odinger equation (and its complex conjugate), we can relate it to $j(\vec r, t)$  

\begin{eqnarray*}
 \frac{\partial |\p|^2}{\partial t} &=& \p\frac{\partial \p^\ast}{\partial t} + \p^\ast\frac{\partial \p}{\partial t} \\
  &=& i\left(\frac{1}{2}\p^\ast\nabla^2\p - V(\vec r)|\p|^2 - \frac{1}{2}\p\nabla^2\p^\ast+ V(\vec r)|\p|^2 \right) \\
  &=& \frac{1}{2}i\Big(\p^\ast\nabla^2\p-\p\nabla^2\p^\ast\Big) \\
  &=& -\nabla\cdot\left(\frac{1}{2i}\left(\p^\ast\nabla\p-\p\nabla\p^\ast\right)\right) \\
  &=& -\nabla\cdot j(\vec r, t)
\end{eqnarray*}

You can argue that the last equality should be a definition, since the equation derived above obviously is the Quantum Mechanical analogue to the continuity equation in classical mechanics\footnote{It should be mentioned that the continuity equation holds as long as the number of particles in the system is conserved \cite{morten}. Chemists often break this condition, e.g. when modeling chemical reactions.}. However, this does not change the fact that inserting the result from Eq.~(\ref{eq:experimentalJ}) yields a diffusion equation

\begin{eqnarray}
 \frac{\partial |\p|^2}{\partial t} &=& -\nabla\cdot j(\vec r, t) \nonumber \\
                                    &\approx& D\nabla^2|\p|^2,
\end{eqnarray}

where the value of the diffusion constant, $D=\frac{1}{2}$, can be extracted from the Scr\"odinger equation \cite{abInitioMC}.

Given that we model the diffusion of walkers by the process described above, the equations themselves serves no practical use. In order to achieve specific sampling rules for our walkers, we need a connection between the time-dependence of the probability distribution and the time-dependence of the configuration space. This connection is given in terms of a stochastic differential equation called \textit{The Langevin Equation}




