\chapter{Quantum Mechanics}

blah blah

\section{Fundamentals}

blah blah

\subsection{The road to Quantum Mechanics}

Ever since mankind first started asking questions about the mechanisms behind the events of the earth, physics has been in development. The revolution of physics however, came in the language of mathematics. As important as the question ``why does the apple fall'', was the question ``at what speed'' and ``at what time''. The framework containing not only explairnations to general phenomenon, but also the tools needed to calculate properties of special events, are often named some type of mechanics. 

For instance, we have the classical mechanics, explaining conservation of evergy, momentum, etc. It's fortresses are Newton's Equations, Lagrange's equations and so on. Written in the language of mathematics, it is extremely effective at describing the motion of systems like i.e. the pendulum. The behaviour calculated lives up to \textit{our expectations} of the system from the ``real-world''.

However, as time passed and classical physics became a well known topic, it's flauds arose. The perhaps most famous of these are the \textit{ultraviolet catastrophe}: The energy in high frequency radioation was simply counter intuitive. If this radiation followed classical electrodynamics, we would all be fried centuries ago. Something was off. Max Planch served the solution: Energy is \textit{quantized}. You do not have a continous energy spectre in a system. Some energies are high, some are low, but they are not continiously connected as in classical physics. This time classical physics did not do well, but it couldn't be all wrong! It serves perfect solutions to systems like the pendulum mentioned above; it lives up to our expectations at some level.

So what does expectations have to do with anything? And what's this level where we cannot trust it anymore? Let us assume that Newton's second law only holds for expectation values of quantities:


$$ \vec F = m\vec a \Longrightarrow \langle F \rangle =  \langle ma \rangle $$
$$\vec F = -\vec\nabla V\,\,\,\,\,\,\,\,\,\,\,\,\, m\vec a = \frac{d\vec p}{dt}$$
$$\langle -\vec\nabla V \rangle =  \langle \frac{d\vec p}{dt} \rangle. $$

So, whatever more complex mathematical framework lies behind our classical physics, it should obey this relation. It must produce what we expect from the real world. ``Changing'' the fundamental theory should not make the pendulum swing any different. As stated by Ehrenfest in his theorem:

\vspace{0.5cm}
\textit{Expectation values of variables follows classical paths.}
\vspace{0.5cm}

Thinking of it, if only some of the gazillions of particles making up the pendulum behaves non-classical, it doesn't make a difference. It's neglectable in the bigger picture, since most of them will behave classicly. In other words: When you throw an apple, you don't throw one object, you throw trillions of atoms.

So, what's the underlying equation which, when taken the expectationvalue of, becomes the equation derived from Newton's second law above? It's the fortress of Quantum Mechanics; \textit{the Schr�dinger Equation}:

$$i\hbar\frac{\partial\psi(r, t)}{\partial t} = -\frac{\hbar^2}{2m}\nabla^2\psi(r, t) + V(r)\psi(r, t)$$

The $\psi(r, t)$ is called the \textit{wavefunction}. It's the goal of the computation. Once you have the wavefunction, you have everything. It acts as the probability amplitude of the system. The probability distribution function is the square of the amplitude:

$$P(r, t) = \left|\psi(r, t)\right|^2.$$

This basically means that the outcome of a measurement is not given in forehand. All you can calculate is the probability of measuring a certain value (remember that values, like i.e. energy, is quantized), not what one measurement will yield. This, however, is a mathematical formulation. What makes up the physics is how we interpret it. Either, the measurement is indeed given in forehand, we just do not have the rights to know, or, on the other hand, we follow Niels Bohr's interpretation, and say that the wavefunction is a superposition of all possible states with different weights, and a measurement will simply collapse it into one of them.