\chapter{Introduction}

Studies of general systems demand a general solver. The process of developing code aimed at a specific task is fundamentally different from the process of developing a general solver, simply due to the fact that the general equations need to be implemented \textit{independent} of any specific properties a modelled system may contain. This is most commonly achieved through object oriented programming, which allows for the code to be structured into general implementations and specific implementations. The general - and specific implementations can then be interfaced through a functionality referred to as \textit{polymorphism}. The aim of this thesis is to use object oriented C++ to build a general and efficient Quantum Monte-Carlo (QMC) solver, which can tackle several many-body systems, from confined electron systems, i.e. quantum dots, to bosons.  

A constraint put on the QMC solver in this thesis is that the ansatz for the \textit{trial wave function} consists of a single term, i.e. a single \textit{Slater determinant}. This opens up the possibility to study systems consisting of a large number of particles, due to efficient optimizations in the single determinant. A simple trial wave function will also  significantly ease the implementation of different systems, and thus make it easier to develop a general framework within the given time frame.

Given the simple ansatz for the wave function, the precision of Variational Monte-Carlo (VMC) is expected to be far from optimal, however, Diffusion Monte-Carlo (DMC) is supposed to withstand this problem, and thus yield a good final estimate nevertheless. To study this purposed power of DMC is another main focus of this thesis, in addition to pushing the limits regarding optimization of the code, and thus run ab-initio simulations of a large number of particles.

The two-dimensional quantum dot was chosen as the system of reference around which the code was planned. The reason for this is that all the current Master students are studying quantum dots at some level, which means that we can help each other reach a collective understanding of the system. Additionally, Sarah Reimann has studied two-dimensional quantum dots for up to 56 particles using a non-variational method called \textit{Similarity Renormalization Group theory} \cite{verdensBesteArtikkel}. Providing her with precise variational DMC benchmarks were considered to be of utmost importance. Coupled Cluster Singles and Doubles (CCSD) results are done up to 56 particles by Christoffer Hirth \cite{Hirth}, however, for the lower frequencies, i.e. for higher correlations, CCSD struggles with convergence.   

Depending on the success of the implementation, various additional systems could be implemented and studied in detail, such as atomic systems, three-dimensional - and double-well quantum dots. 

Apart from benchmarking DMC ground state energies, the specific aim in the case of quantum dots is to study their behavior as the frequency is lowered. A lower frequency implies a higher correlation in the system. Understanding these correlated systems of electrons are of great importance to many-body theory in general. The effect of adding a third dimension is also of high interest. The advantage of DMC compared to other methods is that the distribution is relatively easy to obtain.

Ground state energies for atomic systems can be benchmarked against experimental results \cite{H_He_exact, ExactMolecules, AtomsExact, KryptonExact}, that is, precise calculations which are believed to be very close to the exact result for the given Hamiltonian, which yields an excellent opportunity to test the limits of DMC given a simple trial wave function. Going further to molecular systems, an additional aim is to explore the transition between QMC and molecular dynamics by parameterizing simple force field potentials \cite{forcesQMC}.

Several former Master students, such as Christoffer Hirth \cite{Hirth} and Veronica K.B. Olsen \cite{Olsen}, have studied two-dimensional quantum dots in the past, and have thus generated ground state energies to  which the DMC energies can be compared. For three-dimensional quantum dots, few results are available for benchmarking.

\subsubsection{The structure of the thesis}

\begin{itemize}
 \item The first chapter introduces the concept of object oriented programming, with focus on the methods used to develop the code for this thesis. The reader is assumed to have some background in programming, hence the very fundamentals of programming are not presented. A full documentation of the code is available in Ref. \cite{libBorealisCode}. The code will thus not be covered in full detail. In addition to concepts from C++ programming, Python scripting will be introduced. General strategies regarding planning and structuring of code will also be covered in detail. The two most important Python scripts used in this thesis are documented in Appendix \ref{appendix:sympy} and Appendix \ref{appendix:DCVIZ}.
 
 \item The second chapter serves as a theoretical introduction to QMC, discussing the necessary many-body theory in detail. Important theory which is required to understand the concepts introduced in later chapters are given the primary focus. The reader is assumed to have a basic understanding of Quantum Wave Mechanics. An introduction to the commonly used Dirac notation is given in Appendix \ref{app:Dirac}.
 
 \item Chapter \ref{ch:optAndGen} presents all the assumptions regarding the systems modelled in this thesis together with the aims regarding the generalization and optimization of the code. The strategies applied to achieve these aims will then be covered in high detail.
 
 \item Chapter \ref{ch:modelledSystems} introduces the systems modelled in this thesis, that is, the quantum dots and atomic systems. The single particle wave functions used to generate the trial wave functions for the different systems are presented together with the respective Hamiltonians.
 
 \item The results, along with the discussions and the conclusions mark the final part of this thesis. Results for up to 56 electrons in the two-dimensional quantum dot are presented and comparisons are made with two - and three-dimensional quantum dots for high and low frequency ranges. A brief display of a double-well quantum dot is then given before the atomic results are presented. The ground state energies of atoms up to Krypton and molecules up to $\mathrm{O_2}$ are then compared to experimental values. Concluding the results section, the molecular energies as a function of the separation of cores are compared to the Lennard-Jones 12-6 potential \cite{MD1, MD2}. Final remarks are then made regarding further work expanding on the work done in this thesis. 
\end{itemize}