\chapter{Introduction}

Studies of general systems requires a general solver. The process of developing code aimed at a specific task is fundamentally different from the process of developing a general solver, simply due to the fact that the general equations need to be implemented \textit{independent} of any specific properties a modelled system may contain. This is most commonly achieved through object oriented programming, which allows for the code to be structured into general implementations and specific implementations. Specific - and general implementations can be interfaced without specifying the specific parts through a functionality referred to as \textit{polymorphism}. This is the strategy used in this thesis to achieve a general Quantum Monte-Carlo (QMC) solver.  

The most restrictive constraint put on the QMC solver is that the ansatz for the \textit{trial wave function} consists of a single term, i.e. a single \textit{Slater determinant}. This opens up the possibility to study systems consisting of a large number of particles, due to efficient optimizations in the single determinant. Variational Monte-Carlo (VMC) will suffer due to the fact that the ansatz is simple, however,  Diffusion Monte-Carlo (DMC) is supposed to withstand the problems introduced by a simplistic trial wave function and thus yield a good final estimate nevertheless. To study this purposed power of DMC is one of the main focuses of this thesis. A second main focus is to push the limits regarding optimization of the code, and thus run simulations for vast amounts of particles.

The studied systems are focused around systems resembling the well known harmonic oscillator and the well known hydrogen atom. This because the trial wave function can be constructed using the eigenstates of these systems, which serves as a better fit than a randomly selected basis. This is especially important due to the fact that a single determinant ansatz is used. 

What hinders the systems from being solved in a closed form manner is the many-body interactions, which in this thesis is truncated at the level of two-body interactions, i.e. the Coulomb interaction. Electrons confined in the harmonic oscillator potential are referred to as \textit{quantum dots}, whereas the electrons in the ``hydrogen potential'' are referred to as atoms. 

The aim in the case of quantum dots is to study them in both two and three dimensions, looking at similarities and differences as well as their reactions to increased confinement, i.e. a lower frequency. An additional aim is to implement quantum dots with double oscillator well potentials, however, only in order to demonstrate the versatility of the code. For atomic systems, the aim is to implement both general atoms and homonuclear diatomic molecules. The idea with the molecules is to study their energies as a function of the separation of their cores, and compare the result to commonly used models for atomic interactions in molecular dynamics simulations, such as the Lennard-Jones 12-6 potential.  

Experimental ground state energies are available in the case of the atomic systems, which serve as excellent benchmarks for the DMC calculations, however, this is not the case for quantum dots. This is due to the fact that the model used for quantum dots in this thesis is not the same as is used experimentally. Nevertheless, understanding the behavior of strongly confined electrons is of great importance to many-body Quantum Mechanics in general. The studies of quantum dots in this thesis will thus be utterly motivated from an academic point of view. On the topic of benchmarking the results of quantum dots, many former Master students, such as Christoffer Hirth \cite{Hirth} and Veronica K.B. Olsen \cite{Olsen}, have studied the two dimensional system in the past, and has thus generated ground state energies with which the DMC energies can be compared. For three dimensional quantum dots, little results are available for benchmarking.

Apart from ground state energies, the electron densities are easily obtained using QMC compared to other methods such as Coupled Cluster. A goal has thus been to use these densities in discussions regarding the different systems, rather than basing everything on the ground state energies alone.

\subsubsection{The structure of the thesis}

\begin{itemize}
 \item The first chapter introduces the concept of object oriented programming, with focus on the methods used to develop the code for this thesis. The reader is assumed to have some background in programming, hence the fundamentals of programming is assumed well known. The code consists of ~15000 lines of C++, and will thus not be discussed in detail, as this would be bring too much to the table. However, a full documentation of the code is available in Ref. \cite{libBorealisCode}. In addition to concepts from C++ programming, Python scripting will be introduced. General strategies regarding planning and structuring of code will also be covered in detail. Documentation of the two most important Python scripts used in this thesis is given in Appendix \ref{appendix:sympy} and Appendix \ref{appendix:DCVIZ}.
 
 \item The second chapter serves as a theoretical introduction to QMC, discussing the necessary many-body theory in detail. Important theory which is required to understand the concepts introduced in later chapters are given the primary focus. The reader is assumed to have a basic understanding of Quantum wave mechanics. An introduction to the commonly used Dirac notation is given in Appendix \ref{app:Dirac}.
 
 \item Chapter \ref{ch:optAndGen} presents all the assumptions regarding the systems modelled in this thesis together with the aims regarding the generalization and optimization of the code. The strategies applied to achieve these aims will then be covered in high detail.
 
 \item Chapter \ref{ch:modelledSystems} introduces the systems modelled in this thesis. That is, the different quantum dots and atomic systems.
 
 \item The results, along with the discussions and the conclusions mark the final part of this thesis. Results for up to 56 electrons in the two dimensional quantum dot is presented and comparisons are made between two - and three dimensional quantum dots for high and low frequency ranges. A brief display of a double-well quantum dot is then given before the atomic results are presented. The ground state energies of atoms up to Krypton and molecules up to $\mathrm{O_2}$ are then compared to experimental values before the molecular energies as a function of the separation of cores are compared to the Lennard-Jones 12-6 potential. Final remarks are then made regarding further work expanding what was achieved in the thesis. 
\end{itemize}
