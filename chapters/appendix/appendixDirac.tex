\chapter{Dirac Notation}
\label{app:Dirac}

Calculations involving sums over inner products of orthogonal states are common in Quantum Mechanics. This due to the fact that eigenfunctions of Hermitian operators, which is the kind of operators which represent observables\cite{griffiths}, are necessarily orthogonal\cite{golub1996matrix}. These inner-products will in many cases result in either zero or one, i.e. the \textit{Kronecker-delta} function $\delta_{ij}$; explicitly calculating the integrals is unnecessary. 

\textit{Dirac notation} is a notation in which quantum states are represented as abstract components of a \textit{Hilbert space}, i.e. an inner product space. This implies that the inner-product between two states are represented by these states alone, without the integral over a specific basis, which makes derivations a lot cleaner and general in the sense that no specific basis is needed.

Extracting the abstract state from a wave function is done by realizing that the wave function can be written as the inner product between the position basis eigenstates $\ket{x}$ and the abstract quantum state $\ket{\psi}$ 

\begin{equation*}
 \psi(x) = \langle r, \psi \rangle \equiv \braket{x}{\psi} = \bra{x}\times\ket{\psi}. 
\end{equation*}


The notation is designed to be simple. The right hand side of the inner product is called a \textit{ket}, while the left hand side is called a \textit{bra}. Combining both of them leaves you with an inner product bracket, hence Dirac notation is commonly referred to as \textit{bra-ket} notation. 

To demonstrate the simplicity introduced with this notation, imagine a coupled two-level spin-$\frac{1}{2}$ system in the following state

\begin{eqnarray}
 \ket{\chi} &=& N\Big[\ket{\su\sd} -i\ket{\sd\su}\Big]\\
 \bra{\chi} &=& N\Big[\bra{\su\sd} +i\bra{\sd\su}\Big]
\end{eqnarray} 

Using the fact that both the $\ket{\chi}$ state and the two-level spin states should be orthonormal, the normalization factor can be calculated without explicitly setting up any integrals

\begin{eqnarray*}
 \braket{\chi}{\chi} &=& N^2\Big[\bra{\su\sd} +i\bra{\sd\su}\Big]\Big[\ket{\su\sd} -i\ket{\sd\su}\Big] \\
 &=& N^2\Big[\braket{\su\sd}{\su\sd} + i\braket{\sd\su}{\su\sd} - i\braket{\su\sd}{\sd\su} + \braket{\sd\su}{\sd\su}\Big] \\
 &=& N^2\Big[1 + 0 - 0 + 1\Big] \\
 &=& 2N^2\\
 &=& 1,
\end{eqnarray*}

This implies the trivial solution $N=1/\sqrt{2}$. With this powerful notation at hand, important properties such as the \textit{completeness relation} of a set of states can be shown. A standard strategy is to start by expanding one state $\ket{\phi}$ in a complete set of different states $\ket{\psi_i}$:

\begin{eqnarray*}
 \ket{\phi}            &=& \displaystyle\sum_i c_i\ket{\psi_i}\\
 \braket{\psi_k}{\phi} &=& \displaystyle\sum_i c_i\underbrace{\braket{\psi_k}{\psi_i}}_{\delta_{ik}}\\
                       &=& c_k\\
 \ket{\phi}            &=& \displaystyle\sum_i \braket{\psi_i}{\phi}\ket{\psi_i}\\
                       &=& \left[\displaystyle\sum_i \ket{\psi_i}\bra{\psi_i}\right]\ket{\phi} 
\end{eqnarray*}

which implies that

\begin{equation}
 \displaystyle\sum_i \ket{\psi_i}\bra{\psi_i} = \I
 \label{eq:Completeness}
\end{equation}

for any complete set of orthonormal states $\ket{\psi_i}$. Calculating the corresponding identity for a continuous basis like e.g. the position basis yields

\begin{eqnarray}
 \int |\psi(x)|^2dx    &=& 1 \label{eq:posIdentityFirst} \\
 \int |\psi(x)|^2dx    &=& \int \psi^\ast(x)\psi(x)\mathrm{d}x \nonumber \\
                       &=& \int \braket{\psi}{x}\braket{x}{\psi}\mathrm{d}x \nonumber \\
                       &=& \bra{\psi}\Big[\int \ket{x}\bra{x}dx\Big]\ket{\psi} \label{eq:posIdentityLast}.
\end{eqnarray}

Combining eq.~\ref{eq:posIdentityFirst} and eq.~\ref{eq:posIdentityLast} with the fact that $\braket{\psi}{\psi} = 1$ yields the identity

\begin{equation}
 \int \ket{x}\bra{x}dx = \I.
 \label{eq:CompletenessCont}
\end{equation}

Looking back at the introductory example, this identity is exactly what is extracted when a wave function is described as an inner product instead of an explicit function.
