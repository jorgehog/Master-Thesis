\chapter{Dirac Notation}

Due to the orthogonal nature of Hermitian operators' eigenfunctions\footnote{Eigenfunctions of a Hermitian operator always make up a complete orthogonal set.}, the inner product between two states constructed from them will result in a lot of integrals being zero, one, or eigenvalues for that matter. Writing the integrals in their full form then feels like a waste of space and time. Even specifying e.g. a position basis is obsolete. Abstracting the wave functions from a given parameter space (e.g. $\mathbb{C}^n$) into a \textit{Hilbert space}\footnote{A Hilbert space is an inner product space spanned by the different states. For every state, there exists a complementary state which is the Hermitian conjugate of the original\cite{griffiths}.} is what is called the \textit{Dirac notation}, or the \textit{Bra-ket notation}.

The basic idea is that since the coordinate representation of a wave function is the projection of an abstract state on the position basis through an inner product, we can separate the different pieces of the inner product:

\begin{equation*}
 \psi(\vec r) = \langle r, \psi_j \rangle \equiv \braket{r}{\psi_j} = \bra{r}\times\ket{\psi_j}. 
\end{equation*}


The notation is designed to be simple. The right hand side of the inner product is called a \textit{ket}, while the left hand side is called a \textit{bra}. Combining both of them leaves you with an inner product bracket, hence the names. Let us look at an example where this notation is extremely powerful. Imagine a coupled two-particle spin-$\frac{1}{2}$ system in the following state

\begin{eqnarray}
 \ket{\psi} &=& N\Big[\ket{\su\sd} -i\ket{\sd\su}\Big]\\
 \bra{\psi} &=& N\Big[\bra{\su\sd} +i\bra{\sd\su}\Big]
\end{eqnarray} 

Using the fact that both the full two-particle state and the two-level spin states should be orthonormal, we can with this notation calculate the normalization factor without explicitly calculating anything.

\begin{eqnarray*}
 \braket{\psi}{\psi} &=& N^2\Big[\bra{\su\sd} +i\bra{\sd\su}\Big]\Big[\ket{\su\sd} -i\ket{\sd\su}\Big] \\
 &=& N^2\Big[\braket{\su\sd}{\su\sd} + i\braket{\sd\su}{\su\sd} - i\braket{\su\sd}{\sd\su} + \braket{\sd\su}{\sd\su}\Big] \\
 &=& N^2\Big[1 + 0 - 0 + 1\Big] \\
 &=& 2N^2
\end{eqnarray*}

This implies as we expected $N=1/\sqrt{2}$. With this powerful notation at hand, we can easily show properties such as the \textit{completeness relation} of a set. We start by expanding one state $\ket{\phi}$ in a complete set of different states $\ket{\psi_i}$:

\begin{eqnarray*}
 \ket{\phi}            &=& \displaystyle\sum_i c_i\ket{\psi_i}\\
 \braket{\psi_k}{\phi} &=& \displaystyle\sum_i c_i\braket{\psi_k}{\psi_i}\\
                       &=& c_k\\
 \ket{\phi}            &=& \displaystyle\sum_i \braket{\psi_i}{\phi}\ket{\psi_i}\\
                       &=& \left[\displaystyle\sum_i \ket{\psi_i}\bra{\psi_i}\right]\ket{\phi},
\end{eqnarray*}

which implies that

\begin{equation}
 \displaystyle\sum_i \ket{\psi_i}\bra{\psi_i} = \I
 \label{eq:Completeness}
\end{equation}

for any complete set of orthonormal states $\ket{\psi_i}$. For a continuous basis like e.g. the position basis we have a a similar relation:

\begin{eqnarray}
 \int |\psi(x)|^2dx    &=& 1 \label{eq:posIdentityFirst} \\
 \int |\psi(x)|^2dx    &=& \int \psi^\ast(x)\psi(x)dx \nonumber \\
                       &=& \int \braket{\psi}{x}\braket{x}{\psi}dx \nonumber \\
                       &=& \bra{\psi}\Big[\int \ket{x}\bra{x}dx\Big]\ket{\psi} \label{eq:posIdentityLast}.
\end{eqnarray}

Combining eq.~\ref{eq:posIdentityFirst} and eq.~\ref{eq:posIdentityLast} with the fact that $\braket{\psi}{\psi} = 1$ yields the identity

\begin{equation}
 \int \ket{x}\bra{x}dx = \I.
 \label{eq:CompletenessCont}
\end{equation}
