\chapter{Visualization of Data}

With an increasingly massive code framework comes the increased need of data analysis tools. Data analysis usually involves visualization of the data; there is only so much information a single number can hold. To supplement the QMC code, a visualization tool has been developed, dramatically easing the implementation of additional visualization scripts.

\section{DCViz}

The tool DCViz (Dynamic Column data Visualizer) is a Python based visualization tool designed to plot data stored in columns. The plot library used is \textit{Matplotlib}. The data can be plotted dynamically at a specified interval, and is hence designed to run parallel to the main application, e.g. DMC. The convergence of the DMC method is much better represented by a trailing energy graph than simply a stream of numbers. The same goes for the minimization process.

\subsection{Implementing a Visualization Tool}

Just like for the \verb+orbitalsGenerator+ tool (see Appendix \ref{appendix:sympy}), specific implementations come in the shape of subclasses of the DCViz superclass. All the functionality regarding setting up figures, re-plotting dynamically, clearing figures to avoid random crashes if used repeatedly, etc. is inherited from the superclass. The necessary elements to implement is

\begin{small}
\begin{tabular}{lp{14cm}}
\verb+figMap+		& A dictionary representing the names and structure of the plotted figures. If e.g. two figures 				\verb+fig1+ and \verb+fig2+ are wanted, where \verb+fig1+ should contain three sub-figures 					\verb+subFig1+, \verb+subFig2+ and \verb+subFig3+, the figure map should be set up as \\					& \verb+figMap = {"fig1": ["subFig1", "subFig2", "subFig3"], "fig2": ["subFig4"]}+ \\
			& In class member functions, e.g. \verb+self.subFig1+ will be available as the figure representing the first sub-figure of the first figure. \\
			\ \\
\verb+nametag+		& A string representing the name of the output files associated with the given implementation 				(subclass). Full regular expressions support. E.g. \verb|nametag = "DMC_out_\d+\.dat"|. All files
			loaded which matches the name-tag will be automatically plotted by the corresponding subclass implementation. \\
			\ \\
\verb+plot(self, data)+& The loaded file will be loaded into a \verb+data+ object, which is an iterator where each elements
			represents a column in the data file. Designed such that expressions such as \verb+col1, col2 = data+ is possible (and fast). For the raw data loaded in a matrix, use \verb+data.data+ to directly access the loaded file.
			\ \\
			&In the plot function, the sub-figures introduced in \verb+figMap+ should be loaded with data through e.g. \verb+self.subFig1.plot(col1, col2)+. The superclass' main loop will take care of the rest.
\end{tabular}
\end{small}

An example implementation would be

\vspace{0.5cm}
\begin{lstlisting}[language=Python]
class myTestClass(DCVizPlotter):
    nametag =  'testcase\d\.dat' #filename with regex support
    
    #1 figure with 1 subfigure
    figMap = {'fig1': ['subfig1']}
    
    #skip first row.
    skipRows = 1    
    
    def plot(self, data):
        column1 = data[0]

        self.subfig1.set_title('I have $\LaTeX$ support!')
              
        self.subfig1.set_ylim([-1,1])
          
        self.subfig1.plot(column1)
\end{lstlisting}



\subsection{Additional Support}

Additional parameters can be overloaded for additional functionality

\begin{small}
\begin{tabular}{lp{14cm}}
\verb+nCols+ & The number of columns present in the file. Will be automatically detected unless the data is stored in binary format.\\
\verb+skipRows+ & The number of initial rows to skip. Will be automatically detected unless the data is stores as a single column.\\
\verb+skipCols+ & The number of initial columns to skip. Defaults to zero.\\
\verb+armaBin+ & Boolean flag. If set to true, the data is assumed to be stored in Armadillo's binary format (doubles). Number of columns and rows will be read from the file header.\\
\verb+fileBin+ & Boolean flag. If set to true, the data is assumed to be stores in binary format. The number of columns must be specified.
\end{tabular}
\end{small}

The \LaTeX support is enabled if the correct packages is installed.

\subsubsection{Families}

A specific implementation can be flagged to belong to a family by setting the class member variable \verb+isFamilyMember+ to true. If this flag is true, the folder where the originally data was loaded will be scanned for additional matches, all of which will be loaded into the \verb+data+ input to the plot function. In this case each element of the \verb+data+ list would be a column iterator as explained previously.

To keep track of which file a given data-set was loaded from, a list \verb+self.familyFileNames+ is created, where element $i$ is the filename corresponding to \verb+data[i]+.

A class member string \verb+familyName+ can be overridden to display a more general name in the auto-detection feedback. An example implementation using data file families would be
\vspace{0.5cm}
\begin{lstlisting}[language=Python]
class myTestClassFamily(DCVizPlotter):
    nametag =  'testcaseFamily\d\.dat' #filename with regex support
    
    #1 figure with 3 subfigures
    figMap = {'fig1': ['subfig1', 'subfig2', 'subfig3']}
    
    #skip first row of each data file.
    skipRows = 1    

    #Using this flag will read all the files matching the nametag
    #(in the same folder.) and make them aviable in the data arg    
    isFamilyMember = True
    familyName = "testcase"
    
    def plot(self, data):
        
        #figures[0] is 'fig1' figures. the 0'th element is the
        #self.fig1 itself. Subfigures are always index [1:]
        mainFig = self.figures[0][0]  
        mainFig.suptitle('I have $\LaTeX$ support!')        
        subfigs = self.figures[0][1:]
    
        #Notice we plot fileData.data (In order to get the numpy object) 
        #and not fileData alone, as fileData is a 'dataGenerator' instance 
        #used to speed up file reading. Alternatively, we could send data[:]
        for subfig, fileData in zip(subfigs, data):
            subfig.plot(fileData.data)
            subfig.set_ylim([-1,1])
\end{lstlisting}

loading e.g. \verb+testcaseFamily0.dat+ would automatically load \verb+testcaseFamily1.dat+ etc. as well.

\subsection{Usage: The API, Terminal Client and GUI}

All listed interfaces to the DCViz core has full warning support and reconfigurability. DCViz, like the QMC code in general, is designed to be used by other people than the Author.

\subsubsection{The Application Programming Interface (API)}
The DCViz library has been developed to interface nicely with any python script \footnote{DCViz can also be called through C++, however, no header has been implemented for this. Typically one would use the std::system or std::thread to start the script in the background.} Given a path to the data file, all that is needed in order to visualize it is


\begin{lstlisting}[language=Python]
import DCVizWrapper as viz
dynamicMode = False #or true

...
#Perform some calculations and store these in the file myDataFile (including path)

#DCVizWrapper.main() automatically detects the subclass implementation 
#matching the specified file. Thread safe and easily interruptable.
viz.main(path=myDataFile, dynamic=dynamicMode)
\end{lstlisting}

\subsubsection{Using the Terminal}

The \verb+DCVizWrapper.py+ script can also be called directly from a terminal using the path as first command line argument. If the option \verb+-d+ is supplied, dynamic mode is activated.

\subsubsection{The GUI}

The script \verb+DCVizGUI.py+ sets up a GUI for visualizing data using DCViz. The GUI is implemented using pyside (python wrapper for QT), and is designed to be simple. Data files are loaded from an open-file dialog, and will appear in a drop-down menu once loaded. The play button executes the main loop of the currently selected data file. Dynamic mode is selected though a check-box, and the refresh interval is set by a slider (from zero to ten seconds). Warnings can be disabled through the configuration file.

The GUI file can be called from another script (threaded) with main path as first commandline arg. 

Ha med screenshot av DCViz i action.





