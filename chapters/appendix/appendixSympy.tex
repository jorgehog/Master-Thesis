\chapter{Auto-generation with SymPy}
\label{appendix:sympy}

``\textit{SymPy is a Python library for symbolic mathematics. It aims to become a full-featured computer algebra system (CAS) while keeping the code as simple as possible in order to be comprehensible and easily extensible. SymPy is written entirely in Python and does not require any external libraries.}'' 

\hfill - The SymPy Home Page \cite{SymPy}

\vspace{0.5cm}
Thie aim of this appendix will be on using SymPy to calculate closed form expressions for single-particle wave functions needed to optimize the calculations of the Slater gradient and Laplacian. For systems of many particles, it is crucial to have these expressions in order for the code to remain efficient. 

Calculating these expressions by hand is a waste of time, given that the complexity of the expressions is proportional to the magnitude of the quantum number, which again scales with the number of particles, and little new insights are gained from doing the calculations. In the case of a $56$ particle Quantum Dot, the number of unique derivatives involved in the simulation is $112$. 

\section{Usage}

SymPy is, as described in the introductory quote, designed to be simple to use. This section will cover the basics needed to calculate gradients and Laplacians, auto-generating C++ - and Latex code.

\subsection{Symbolic Algebra}

In order for SymPy to recognize e.g.~\verb+x+ as a symbol, that is, a \textit{mathematical variable}, special action must be made. In contrast to programming variables, symbols are not initialized to a value. Initializing symbols can be done in several ways, the two most common are listed below

\vspace{0.25cm}
\begin{lstlisting}[language=Python]
In [1]: from sympy import Symbol, symbols

In [2]: x = Symbol('x')

In [3]: y, z = symbols('y z')

In [4]: x*x+y
Out[4]: 'x**2 + y'

\end{lstlisting}

The \verb+Symbol+ function handles single symbols, while \verb+symbols+ can initialize several symbols simultaneously. The string argument might seem redundant, however, this represents the \textit{label} displayed using print functions, which is neat to control. In addition, key word arguments can be sent to the symbol functions, flagging variables as e.g.~positive, real, etc.

\vspace{0.25cm}
\begin{lstlisting}[language=Python]
In [1]: from sympy import Symbol, symbols, im

In [2]: x2 = Symbol('x^2', real=True, positive=True) #Flagged as real. Note the label.

In [3]: y, z = symbols('y z') #Not flagged as real

In [4]: x2+y #x2 is printed more nicely given a describing label
Out[4]: 'x^2 + y'

In [5]: im(z) #Imaginary part cannot be assumed to be anything.
Out[5]: 'im(z)'

In [6]: im(x2) #Flagged as real, the imaginary part is zero.
Out[6]: 0

\end{lstlisting}

\subsection{Exporting C++ and Latex Code}

Exporting code is extremely simple: SymPy functions exist in the \verb+sympy.printing+ module, which simply takes a SymPy expression on input and returns the requested code-style equivalent. Consider the following example

\vspace{0.25cm}
\begin{lstlisting}[language=Python]
In [1]: from sympy import symbols, printing, exp

In [2]: x, x2 = symbols('x x^2')

In [3]: printing.ccode(x*x*x*x*exp(-x2*x))
Out[3]: 'pow(x, 4)*exp(-x*x^2)'

In [4]: printing.ccode(x*x*x*x)
Out[4]: 'pow(x, 4)'

In [5]: print printing.latex(x*x*x*x*exp(-x2))
\frac{x^{4}}{e^{x^{2}}}

\end{lstlisting}

The following expression is the direct output from line five compiled in Latex

\begin{equation*}
 \frac{x^{4}}{e^{x^{2}}}
\end{equation*}

\subsection{Calculating Derivatives}

The $2s$ orbital from hydrogen (not normalized) is chosen as an example for this section

\begin{align}
 \phi_{2s}(\vec r) &= (Zr - 2)e^{-\frac{1}{2}Zr} \\
 r^2 &= x^2 + y^2 + z^2
\end{align}

Calculating the gradients and Laplacian is very simply by using the \verb+sympy.diff+ function

\vspace{0.25cm}
\begin{lstlisting}[language=Python]
In [1]: from sympy import symbols, diff, exp, sqrt

In [2]: x, y, z, Z = symbols('x y z Z')

In [3]: r = sqrt(x*x + y*y + z*z)

In [4]: r
Out[4]: '(x**2 + y**2 + z**2)**(1/2)'

In [5]: phi = (Z*r - 2)*exp(-Z*r/2)

In [6]: phi
Out[6]: '(Z*(x**2 + y**2 + z**2)**(1/2) - 2)*exp(-Z*(x**2 + y**2 + z**2)**(1/2)/2)'

In [7]: diff(phi, x)
Out[7]: '-Z*x*(Z*(x**2 + y**2 + z**2)**(1/2) - 2)*exp(-Z*(x**2 + y**2 + z**2)**(1/2)/2)/(2*(x**2 + y**2 + z**2)**(1/2)) + Z*x*exp(-Z*(x**2 + y**2 + z**2)**(1/2)/2)/(x**2 + y**2 + z**2)**(1/2)'

\end{lstlisting}

Now, this looks like a nightmare. However, SymPy has great support for simplifying expressions through factorization, collecting, substituting etc. The following code demonstrated this quite nicely

\vspace{0.25cm}
\begin{lstlisting}[language=Python]
...

In [6]: phi
Out[6]: '(Z*(x**2 + y**2 + z**2)**(1/2) - 2)*exp(-Z*(x**2 + y**2 + z**2)**(1/2)/2)'

In [7]: from sympy import factor, Symbol, printing

In [8]: R = Symbol('r') #Creates a symbolic equivalent of the mathematical r

In [9]: diff(phi, x).factor() #Factors out common factors
Out[9]: '-Z*x*(Z*(x**2 + y**2 + z**2)**(1/2) - 4)*exp(-Z*(x**2 + y**2 + z**2)**(1/2)/2)/(2*(x**2 + y**2 + z**2)**(1/2))'

In [10]: diff(phi, x).factor().subs(r, R) #replaces (x^2 + y^2 + z^2)^(1/2) with r
Out[10]: '-Z*x*(Z*r - 4)*exp(-Z*r/2)/(2*r)'

In [11]: print printing.latex(diff(phi, x).factor().subs(r, R))
- \frac{Z x \left(Z r -4\right)}{2 r e^{\frac{1}{2} Z r}}

\end{lstlisting}

This version of the expression is much more satisfying to the eye. The output from line 11 compiled in Latex is

\begin{equation*}
 - \frac{Z x \left(Z r -4\right)}{2 r e^{\frac{1}{2} Z r}}
\end{equation*}

SymPy has a general method for simplifying expressions \verb+sympy.simplify+, however, this function is extremely slow and does not behave well on general expressions. SymPy is still young, so nothing can be expected to work perfectly. Moreover, in contrast to \textit{Wolfram Alpha} and \textit{Mathematica}, SymPy is open source, which means that much of the work, if not all of the work, is done by ordinary people on their spare time. The ill behaving simplify function is not really a great loss; full control for a Python programmer is never considered a bad thing, whether it is enforced or not.

\clearpage
Estimating the Laplacian is just a matter of summing double derivatives

\vspace{0.25cm}
\begin{lstlisting}[language=Python]
...

In [12]: (diff(diff(phi, x), x) + 
   ....:  diff(diff(phi, y), y) + 
   ....:  diff(diff(phi, z), z)).factor().subs(r, R)
Out[12]: 'Z*(Z**2*x**2 + Z**2*y**2 + Z**2*z**2 - 10*Z*r + 16)*exp(-Z*r/2)/(4*r)'

In [13]: (diff(diff(phi, x), x) + #Not quite satisfying.
   ....:  diff(diff(phi, y), y) + #Let's collect the 'Z' terms.
   ....:  diff(diff(phi, z), z)).factor().collect(Z).subs(r, R)
Out[13]: 'Z*(Z**2*(x**2 + y**2 + z**2) - 10*Z*r + 16)*exp(-Z*r/2)/(4*r)'

In [14]: (diff(diff(phi, x), x) + #Still not satisfying.
   ....:  diff(diff(phi, y), y) + #The r^2 terms needs to be substituted as well.
   ....:  diff(diff(phi, z), z)).factor().collect(Z).subs(r, R).subs(r**2, R**2)
Out[14]: 'Z*(Z**2*r**2 - 10*Z*r + 16)*exp(-Z*r/2)/(4*r)'

In [15]: (diff(diff(phi, x), x) + #Let's try to factorize once more.
   ....:  diff(diff(phi, y), y) + 
   ....:  diff(diff(phi, z), z)).factor().collect(Z).subs(r, R).subs(r**2, R**2).factor()
Out[15]: 'Z*(Z*r - 8)*(Z*r - 2)*exp(-Z*r/2)/(4*r)'

\end{lstlisting}

Getting the right factorization may come across as tricky, but with minimal training this poses no real problems.

\section{Using the auto-generation Script}

The superclass \verb+orbitalsGenerator+ aims to serve as an interface with the QMC C++ \verb+BasisFunctions+ class, automatically generating the C++ code containing all the implementations of the derivatives for the given single-particle states. The single-particle states are implemented in the generator by subclasses overloading system specific virtual functions which will be described in the following sections.

\subsection{Generating Latex code}

The following methods are user-implemented functions used to calculate the expressions which are in turn automagically converted to Latex code. Once they are implemented, the following code can be executed in order to create the latex output

\vspace{0.25cm}
\begin{lstlisting}[language=Python]
orbitalSet = HO_3D.HOOrbitals3D(N=40) #Creating a 3D harm. osc. object
orbitalSet.closedFormify() 
orbitalSet.TeXToFile(outPath)
\end{lstlisting}

\subsubsection{The constructor}

The superclass constructor takes on input the maximum number of particles for which expressions should be generated and the name of the orbital set, e.g.~\verb+hydrogenic+.
Calling a superclass constructor from a subclass constructor is done in the following way

\vspace{0.25cm}
\begin{lstlisting}[language=Python, otherkeywords={self}]
class hydrogenicOrbitals(orbitalGenerator):
 
    def __init__(self, N):
      
        super(hydrogenicOrbitals, self).__init__(N, "hydrogenic")
        #...
\end{lstlisting}


\subsubsection{makeStateMap}

This function takes care of the mapping of a set of quantum numbers, e.g.~$nlm$ to a specific index $i$. The Python dictionary \verb+self.stateMap+ must be filled with values for every unique set of quantum numbers (not counting spin) in order for the Latex and C++ files to be created successfully. For the three-dimensional harmonic oscillator wave functions, the state map looks like this

\begin{center}
\begin{tabular}{l|ccccccccccccccccccccccc}
$i$   &  0 & & 1 &  2 &  3 & & 4 &  5 &  6 &  7 &  8 &  9 & & 10 &  11 &  12 &  13 &  14 &  15 &  16 &  17 &  18 & 19 \\
\hline
$n_x$ &  0 & & 0 &  0 &  1 & & 0 &  0 &  0 &  1 &  1 &  2 & & 0 &  0 &  0 &  0 &  1 &  1 &  1 &  2 &  2 &  3 \\
$n_y$ &  0 & & 0 &  1 &  0 & & 0 &  1 &  2 &  0 &  1 &  0 & & 0 &  1 &  2 &  3 &  0 &  1 &  2 &  0 &  1 &  0 \\
$n_z$ &  0 & & 1 &  0 &  0 & & 2 &  1 &  0 &  1 &  0 &  0 & & 3 &  2 &  1 &  0 &  2 &  1 &  0 &  1 &  0 &  0 
\end{tabular}
\end{center}

\subsubsection{setUpOrbitals}

Within this function, the orbital elements corresponding to the quantum number mappings made in \verb+makeStateMap+ needs to be implemented in a matching order. The quantum numbers from \verb+self.stateMap+ are calculated prior to this function being called, and can thus be accessed in case they are needed, as is the case for the $n$-dependent exponential factor of the hydrogen-like orbitals.

The $i$'th orbital needs to be implemented in \verb+self.orbitals[i]+, using the \verb+x+, \verb+y+ and \verb+z+ variables defined in the superclass. For the three-dimensional harmonic oscillator, the function is simply

\vspace{0.25cm}
\begin{lstlisting}[language=Python, otherkeywords={self}]
def setupOrbitals(self):
      
    for i, stateMap in self.stateMap.items():
        nx, ny, nz = stateMap
        
        self.orbitals[i] = self.Hx[nx]*self.Hy[ny]*self.Hz[nz]*self.expFactor
  
\end{lstlisting}

where \verb+self.Hx+ and the exponential factor are implemented in the constructor. After the orbitals are created, the gradients and Laplacians cam be calculated by calling the \verb+closedFormify()+ function, however, unless the following member function is implemented, they are going to look messy.

\subsubsection{simplifyLocal}

As demonstrated in the previous example, SymPy expressions are messy when they are fresh out of the derivative functions. Since every system needs to be treated differently when it comes to cleaning up their expressions, this function is available. For hydrogen-like wave functions, the introductory example's strategy can be applied up to the level of neon. Going higher will require more advanced strategies for cleaning up the expressions.

The expression and the corresponding set of quantum numbers are given on input. In addition, there is an input argument \verb+subs+, which if set to false should make the function return the expression in terms of \verb+x+, \verb+y+ and \verb+z+ without substituting e.g.~$x^2 + y^2 = r^2$.

\subsubsection{genericFactor}

The method serves as convenient function for describing generic parts of the expressions, e.g.~the exponentials, which are often reused. A set of quantum numbers are supplied on input in case the generic expression depends on these. In addition, a flag \verb+basic+ is supplied on input, which if set to true should, as in the simplify function, return the generic factor in Cartesian coordinates. This generic factor can then easily be taken out of the Latex expressions and mentioned in the caption in order to clean up the expression tables.

\subsubsection{\_\_str\_\_}

This method is invoked by calling \verb+str(obj)+ on an arbitrary Python object \verb+obj+. In the case of the orbital generator class, this string will serve as an introductory text to the latex output.

\subsection{Generating C++ code}

A class \verb+CPPbasis+ is constructed to supplement the orbitals generator class. This objects holds the empty shells of the C++ constructors and implementations. After the functions described in this section are implemented, the following code can be executed to generate the C++ files

\vspace{0.25cm}
\begin{lstlisting}[language=Python]
orbitalSet = HO_3D.HOOrbitals3D(N=40) #Creating a 3D harm. osc. object
orbitalSet.closedFormify() 
orbitalSet.TeXToFile(outPath)
orbitalSet.CPPToFile(outPath)
\end{lstlisting}


\subsubsection{initCPPbasis}

Sets up the variables in the \verb+CPPbasis+ object needed in order to construct the C++ file, such as the dimension, the name, the constructor input variables and the C++ class members. The following function is the implementation for the two-dimensional harmonic oscillator

\vspace{0.25cm}
\begin{lstlisting}[language=Python, otherkeywords={self}]
def initCPPbasis(self):
    
    self.cppBasis.dim = 2
    
    self.cppBasis.setName(self.name)
    
    self.cppBasis.setConstVars('double* k',          #sqrt(k2)
                               'double* k2',         #scaled oscillator freq.
                               'double* exp_factor') #The exponential
                               
    self.cppBasis.setMembers('double* k', 
                             'double* k2',         
                             'double* exp_factor',  
                             'double H',             #The Hermite polynomial part
                             'double x',
                             'double y',
                             'double x2',            #Squared Cartesian coordinates
                             'double y2') 
\end{lstlisting}

\subsubsection{getCPre and getCreturn}

The empty shell of the \verb+BasisFunctions::eval+ functions in the \verb+CPPbasis+ class is implemented as below

\vspace{0.5cm}
\begin{lstlisting}[otherkeywords={self}]
self.evalShell = """
double __name__::eval(const Walker* walker, int i) {

__necessities__
    
    //__simpleExpr__
    
__preCalc__
    return __return__
    
}
"""
\end{lstlisting}

where \verb+__preCalc__+ is a generated C++ expression returned from \verb+getCpre()+, and \verb+__return__+ is the returned C++ expression from \verb+getCreturn()+. The commented \verb+__simpleExpr__+ will be replaced by the expression in nicely formatted SymPy output code. \verb+__necessities__+ is automatically detected by the script, and represents the Cartesian variable expressions needed by the expressions.

The functions take a SymPy generated expression on input, i.e.~an orbital, gradient or Laplacian, and the corresponding index of the expression $i$. The reason these functions are split into a precalculation and a return expression is purely cosmetic. Consider the following example output for the hydrogen-like wave functions:

\vspace{0.5cm}
\begin{lstlisting}
double dell_hydrogenic_9_y::eval(const Walker* walker, int i) {

    y = walker->r(i, 1);
    z = walker->r(i, 2);

    z2 = z*z;
    
    //-y*(k*(-r^2 + 3*z^2) + 6*r)*exp(-k*r/3)/(3*r)
    
    psi = -y*((*k)*(-walker->get_r_i2(i) + 3*z2) + 6*walker->get_r_i(i))/(3*walker->get_r_i(i));
    return psi*(*exp_factor);
    
}
\end{lstlisting}

The \verb+*exp_factor+ is the precalculated $n=3$ exponential which is then simply multiplied by the non-exponential terms before being returned. The commented line is a clean version of the full expression. The required Cartesian components are retrieved prior to the evaluation.

The full implementation of \verb+getCpre()+ and \verb+getCreturn()+ for the hydrogen-like wave functions are given below

\vspace{0.5cm}
\begin{lstlisting}[language=Python, otherkeywords={self}]
def getCReturn(self, expr, i):
     return "psi*(*exp_factor);" 
     
def getCPre(self, expr, i):
     qNums = self.stateMap[i]
     return "    psi = %s;" % printing.ccode(expr/self.genericFactor(qNums))
\end{lstlisting}



\subsubsection{makeOrbConstArg}

Loading the generated \verb+BasisFunctions+ objects into the \verb+Orbitals+ object in the QMC code is rather a dull job, and is not designed to be done manually. The function \verb+makeOrbConstArg+ is designed to automate this process. This is best demonstrated by an example: Consider the following constructor of the hydrogen-like wave function's orbital class

\vspace{0.5cm}
\begin{lstlisting}
basis_functions[0] = new hydrogenic_0(k, k2, exp_factor_n1);
basis_functions[1] = new hydrogenic_1(k, k2, exp_factor_n2);
//...
basis_functions[5] = new hydrogenic_5(k, k2, exp_factor_n3);
//..
basis_functions[14] = new hydrogenic_14(k, k2, exp_factor_n4);
//..
basis_functions[17] = new hydrogenic_17(k, k2, exp_factor_n4);
\end{lstlisting}

where \verb+exp_factor_nk+ represents $\exp(-Zr/k)$, which is saved as a pointer reference for reasons explained in Section \ref{sec:optSPWFqnumIndie}. The same procedure is applied to the gradients and the Laplacians as well, leaving a total of 90 sequential initializations. Everything needed in order to auto-generate the code is the following implementation

\vspace{0.5cm}
\begin{lstlisting}[language=Python, otherkeywords={self}]
def makeOrbConstArgs(self, args, i):
    n = self.stateMap[i][0]
    args = args.replace('exp_factor', 'exp_fa-ctor_n%d' % n)
    return args
\end{lstlisting}

which ensures that the input arguments to e.g.~element $1$ is \verb+(k, k2, exp_factor_n2)+, since the single-particle orbital \verb+self.phi[1]+ has a principle quantum number $n=2$. The input argument \verb+args+ is the default constructor arguments set up the the \verb+initCPPbasis+, and is in the case of hydrogen-like wave functions \verb+(k, k2, exp_factor)+.

The tables listed in Appendix \ref{appendix:SymPyHO}, \ref{appendix:SymPyHO3D} and \ref{appendix:SymPyHydro} are all generated within seconds using this framework. The generated C++ code for these span 8975 lines not counting blank ones.