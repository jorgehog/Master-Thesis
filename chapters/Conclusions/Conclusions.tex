\chapter{Conclusions}

The focus of the thesis was do develop an optimized yet general Quantum Monte-Carlo (QMC) solver which could simulate systems with a large number of particles. This was achieved by using object oriented C++, resulting in $~15000$ lines of code. The code was redesigned four times before the final structure was implemented. The decisions made regarding the final structure were carefully planned in advance of the coding process. 

In order to maintain efficiency for a high number of particles, a series of optimizations were implemented. These optimizations were a result of a thorough profiling of the code. The profiling revealed which parts of the code that took the most time. These parts were one by one optimized. Simulations for a high number of particles would not have been possible without closed form expressions for the derivatives of the single particle wave functions, which was successfully generated by using SymPy. This task is described in detail in Appendix \ref{appendix:sympy}. A total of 252 expressions were generated.

Several Master projects over the years have involved QMC simulations of some sort, like Variational Monte-Carlo (VMC) calculations of two dimensional quantum dots up to 42 particles \cite{larseivind}, and VMC calculations of atoms up to Silicon \cite{vmcAtoms}. Taking the step further to full Diffusion Monte-Carlo (DMC) studies of both systems, adding three dimensional - and double-well quantum dots in addition to homonuclear diatomic molecules and increasing the system sizes to 56 particles and Krypton respectively, serves as a natural extension to these projects.   

The implementation of the visualization tool described in Appendix \ref{appendix:DCVIZ} made simulating a vast amount of different systems possible without getting lost in the enormous amount of data.

Over 100 degrees of freedom were simulated simultaneously with DMC for both two dimensional quantum dots and atoms. Moreover, double-well quantum dots and molecules were implemented by adding $~200$ lines of code.

When it comes to the specific results, the DMC results for two dimensional quantum dots were in excellent agreement with already existing sources for all ranges of frequencies. The near exact two particle results from Full Configuration \cite{Olsen} were reproduced to five decimal precision, while the 56 particle result were lower than the corresponding ones computed with Coupled Cluster Singles and Doubles (CCSD) \cite{Hirth}, which is considered more exact due to the fact that both DMC and CCSD are variational methods. 

