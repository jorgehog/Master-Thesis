\chapter{Conclusions}

The focus of this thesis was to develop an efficient and general Quantum Monte-Carlo (QMC) solver which could simulate various systems with a large number of particles. This was achieved by using object oriented C++, resulting in $\sim15000$ lines of code. The code was redesigned a total of four times. The final structure was carefully planned in advance of the coding process.

It became apparent that in order to maintain efficiency for a high number of particles, closed form expressions for the single particle wave function derivatives were necessary. For two-dimensional quantum dots, 112 of these expressions were needed to simulate the 56-particle system. Needless to say, this process had to be automated. This was achieved by using \textit{SymPy}, an open source symbolic algebra package for Python, wrapped in a script which generated all the C++ necessary code within seconds. This task is described in detail in Appendix \ref{appendix:sympy}. A total of 252 expressions were generated.

As the solver became more and more flexible, the dire need of a control script arose. Hence the current state of the code is 100\% controlled by a Python script. The script translates configuration files to instructions which are fed to the pre-compiled C++ program. In addition, the script sets up the proper environment for the simulation, that is, it makes sure the simulation output is stored in a folder stamped with the current date and time, as well as with a supplied name tag. All in all, this made it possible to run an immense amount of different simulations without loosing track of the data.

In order to handle the increase in data, a script which automatically converted the output files to Latex tables were written. However, this would not uncover whether or not the simulations had converged. An additional script was thus written, which made the process of visualizing immense amounts of data very simple. Moreover, the data could be analyzed real-time, which meant that failing simulations could be detected and aborted before they were complete, saving a lot of time. Expanding the script to handle new output files was by design unproblematic. This script is covered in high detail in Appendix \ref{appendix:DCVIZ}.

Several Master projects over the years have involved QMC simulations of some sort, like Variational Monte-Carlo (VMC) calculations of two-dimensional quantum dots up to 42 particles \cite{larseivind}, and VMC calculations of atoms up to Silicon (14 particles) \cite{vmcAtoms}. In this thesis, the step was taken to full Diffusion Monte-Carlo (DMC) studies of both systems, adding three-dimensional - and double-well quantum dots in addition to homonuclear diatomic molecules. Additionally, the simulation sizes were increased to 56 electrons and Krypton (36 particles) for two-dimensional quantum dots and atoms, respectively.    

The optimization of the code was done by profiling the code, focusing on the parts which took the most time. Due to the general structure of the code, the function responsible for diffusing the particles was responsible for almost all of the runtime. This function is at the core of Adaptive Stochastic Gradient Descent (ASGD), VMC and DMC. In other words, the task of optimization the full code decreased down to the task of optimizing this specific function. By optimizing one part of the diffusion process at the time, the final runtime was successfully reduced to $5\%$ of the original.

Having successfully implemented five different systems demonstrates the code's generality. The implementation of molecules and the double-well was done by adding no more than 200 lines of code. Additionally,  implementing three-dimensional quantum dots was done in an afternoon. Overall the code has lived up to every single one of the initial aims, with the exception of simulating bosons. Studying new fermionic systems were considered more interesting, hence specific implementations for bosons were abandoned.  

For quantum dots, both VMC and DMC perform very well. The results from Full Configuration Interaction \cite{Olsen} for two particles, which are believed to be very close to the exact solution, are reproduced to five digits using DMC, with the VMC result being a little higher (2-3 digits). For atomic systems, the difference in results in VMC and DMC increase. This is expected since the trial wave function does not include the unbound states of the atoms. Nevertheless, DMC performs very well, reproducing the experimental results for two particles with an error at the level of $10^{-4}$. For heavier atoms, the error increases somewhat, however, taking into consideration the simple trial wave function, the results are remarkably good.

Moreover, it was found that the radial distributions of two - and three-dimensional quantum dots with the same number of closed shells were remarkably similar. This, however, is only the case at high frequencies. For lower frequencies, both systems were found to transition into Wigner crystallized states, however, the distributions were no longer similar. This breaking of symmetry is an extremely interesting phenomenon, which can be studied in greater detail in order to say something general about how the number of spatial dimensions affect systems of confined electrons. The Wigner crystallization is covered in high detail in the literature \cite{WignerTransport, WignerPathTo, WignerSymmetryBreak, WignerFloating, Wigner2DQD}, however, a new approach using the virial theorem was investigated in order to describe the transition.  

It is clear that the energy of $\mathrm{H_2}$ graphed as a function of the core separation resembles the Lennard-Jones 12-6 potential. This demonstrates that the code can be used to parameterize potential energies for use in molecular dynamics simulations, however, to produce realistic potentials, support for more complicated molecules must be implemented. 

\subsubsection{Prospects and future work}

Shortly after handing in this thesis, I will focus my effort on studying the relationship between two - and three-dimensional quantum dots in higher detail. The goal is to publish my findings, and possibly say something general regarding how the number of dimensions affect a system of confined electrons.

Additionally, the double-well quantum dot will be studied in higher detail using realistic values for the parameters. These results can then be benchmarked with the results of Sigve Bøe Skattum and his \textit{Multi-configuration Time Dependent Hartree-Fock} solver \cite{Sigve}.

My supervisor and I will at the same time work with implementing a momentum space version of QMC. This has the potential of describing nuclear interactions in great detail \cite{momentspaceQMC}. 

I will continue my academic career as a PhD student in the field of \textit{multi-scale physics}. The transition from QMC to molecular dynamics will thus be of high interest. The plan is to expand the code to general molecules. However, in order to maintain a reasonable precision, the single particle wave functions need to be optimized. Hence implementing a Hartree-Fock solver \cite{Shavitt} or using a Coupled Cluster wave function \cite{CCSD_WF} will be prioritized.


