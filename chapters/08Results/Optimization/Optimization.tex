\section{Optimization Results}

The optimization results listed in this section are estimated using a $30$ particle two-dimensional quantum dot as reference system.

Profiling the code revealed, not surprisingly, that $~99\%$ of the runtime was spent diffusing the particles, i.e. spent in \verb+QMC::diffuse_walker+, regardless of whether VMC, DMC or ASGD was run. Optimizing the code then solely involved optimizing this function. 

The profiling tool of choice was \textit{KCacheGrind}, available at the Ubuntu Software Center. KCacheGrind lists relative time spent in functions graphically in blocks, whose size is proportional to the time spent inside the function, much like standard hard drive listing software does with files and file size.

Optimizations discussed in chapter \ref{ch:optAndGen} which are not mentioned in the following sections are optimizations implemented prior to the optimization process. These optimizations were considered ``standard optimizations'' and was thus implemented from start. 

\subsubsection{Storing the Slater Matrix}

This optimization is described in detail in Section \ref{sec:storeSlater}. In addition to storing the slater, the calculation of $\tilde I$ from the Slater inverse updating algorithm in Eq.~(\ref{eq:Itilde}) was pre calculated outside the main loops.

The percentages listed in the following table is the total time spent inside this specific function relative to all other functions. 

\begin{tabular}{ll}
 \verb+Orbitals::phi+ & \\
 \hline\hline
 Relative runtime used prior to optimization & 80.88\% \\
 Relative runtime used after optimization    & 8.2\% \\
 \hline
 Relative function speedup                   & 9.86
\end{tabular}

The speedup is not a result of optimizations within the function itself, but a result of far less calls to the function. If $\tilde I$ was calculated outside the main loops in the first place, the speedup would be far less significant. 


\subsubsection{Optimized Jastrow Gradient}

The optimization described in this Section is discussed in detail in Section \ref{sec:optJastGrad}.

The percentages listed in the following table is the total time spent inside this specific function relative to all other functions. 

\begin{tabular}{ll}
 \verb+Jastrow::get_grad+ \& \verb+Jastrow::calc_dJ+ & \\
 \hline\hline
 Relative runtime used prior to optimization & 40\% \\
 Relative runtime used after optimization    & 5.24\% \\
 \hline
 Relative function speedup                   & 7.63
\end{tabular}

Exploiting the symmetries of the Padé Jastrow Gradient, in addition to calculating the new gradient based on the old, is in other words extremely efficient. Keep in mind however, that these results are for a high number of particles. For e.g. two particles, this optimization would not matter at all (they actually slow down the code).

\subsubsection{Storing the Orbital Derivatives}

This optimization is covered in detail in Section \ref{sec:storeSlater}. Much like for the Slater matrix, the optimization in this case comes from the fact that the function itself is called fewer times, rather than being optimized.

The percentages listed in the following table is the total time spent inside this specific function relative to all other functions. 


\begin{tabular}{ll}
 \verb+Orbitals.dell_phi+ & \\
 \hline\hline
 Relative runtime used prior to optimization & 56.27\% \\
 Relative runtime used after optimization    & 7.31\% \\
 \hline
 Relative function speedup                   & 7.70
\end{tabular}


\subsubsection{Storing the Quantum Number Independent Terms}

This optimization is covered in detail in Section \ref{sec:optSPWFqnumIndie}. This optimization lowers the number of exponential function calls, and hence optimizes the calculation time of single particle states, its gradients and Laplacians.

The percentages listed in the following table is the total time spent inside this specific function relative to all other functions. 

\begin{tabular}{ll}
 \verb+Orbitals::phi+ \& \verb+Orbitals::dell_phi+ & \\
 \hline\hline
 Relative runtime used prior to optimization & 5.85\% \\
 Relative runtime used after optimization    & 0.13\% \\
 \hline
 Relative function speedup                   & ~45
\end{tabular}

This result is not surprisingly equal to $15\cdot 3$, since a $30$ particle quantum dot has $15$ unique quantum numbers. One set is used by the orbitals, and two by their gradients (the Laplacian is not a part of the diffusion). Prior to this optimization, $45$ exponential calls was needed to fill a row in the Slater matrix and the derivative matrix; this has been reduced to one.

\subsubsection{Overall Optimization}

Combining all the optimizations listed in this chapter, the final runtime was reduced to $5\%$ of the original. The final scaling are presented in the following figures

\textbf{TA MED FIGURER FOR SKALERING AV KODE}