\section{Quantum Dots}

The focus regarding quantum dots has been on studying the distribution of electrons as a function of the level of confinement. In addition, ground state energies are provided and compared to other many-body methods to demonstrate the efficiency and precision of Variational Monte-Carlo (VMC) and Diffusion Monte-Carlo (DMC). In the case of two-dimensional quantum dots, there are multiple published results, however, for three dimensions this is not the case. An introduction to quantum dots is given in Section \ref{sec:modelQDots}. 

The double-well quantum dot has not been a focus in this thesis, however, some simple results are provided to demonstrate the flexibility of the code.

\subsection{Ground State Energies}

\subsubsection{two-dimensional quantum dots}

Table \ref{tab:QDotsResultsAll} presents the calculated ground state energies for two-dimensional quantum dots in addition to corresponding results from methods such as Similarity Renormalization Group theory (SRG), Coupled Cluster Singles and Doubles (CCSD) and Full Configuration Interaction (FCI). In addition, some previously published DMC results are supplied. The references are listed in the table caption.

In light of the variational nature of DMC and VMC, the results show that DMC provides a more precise estimate for the ground state energy than VMC, both in terms of lower energies and lower errors. The exact energy in the case of two electrons with $\omega=1$ has been calculated in Ref. \cite{taut} and is $E_0 = 3$, which is in excellent agreement with the presented results.

The statistical errors in the DMC energies calculated in this thesis are lower than those provided in Ref. \cite{MagnusArticle}. This may only be due to the fact that the calculations in this thesis have been run on a super computer. Running smaller simulations on fewer processors result in larger errors. Both the implementations successfully agree with the FCI result for two particles, which strongly indicates that the disagreements in results are a result of systematic errors.    

In the case of two particles, DMC and FCI agree up to five decimals, which leads to the conclusion that DMC indeed is a very precise method. The SRG method is not variational in the sense that it can undershoot the exact energy. The DMC result should thus not be read as less precise in the cases where SRG provides a lower energy estimate. Diffusion Monte-Carlo and SRG are in excellent agreement for a large number of particles compared to FCI and CCSD, which drift away from the DMC results as their basis sizes shrink. 

For high frequencies, the VMC energy is higher than the CCSD energy. The fact that both the methods are variational implies that CCSD performs better than VMC in this frequency range. However, looking at the results the lower frequency range, it is clear that VMC performs better than CCSD. This is due to the fact that CCSD struggles with convergence as the correlations within the system increase, indicated by the decrease in number of shells used to perform the calculations.

The DMC energy is overall smaller than the CCSD energy, which, due to the variational nature of the methods, implies that DMC performs better than CCSD. Nevertheless, the results for $56$ particles are in excellent agreement. 

\setlength{\tabcolsep}{5pt}
\begin{table}
\begin{center}
\begin{tabular}{cc|rrrrrr}
    N     & $\omega$ & $\mathrm{E_{VMC}}$ & $\mathrm{E_{DMC}}$ & $E_\mathrm{ref}^{(a)}$& $E_\mathrm{ref}^{(b)}$ & $E_\mathrm{ref}^{(c)}$ & $E_\mathrm{ref}^{(d)}$\\
\hline\hline
\multicolumn{8}{c}{} \\
    2     &   0.01   & 0.07406(5)  & 0.073839(2)  & -		& -			& 0.0738 \{23\} & 0.07383505 \{19\}\\
          &   0.1    & 0.44130(5)  & 0.44079(1)   & - 		& - 			& 0.4408 \{23\} & 0.44079191 \{19\}\\
          &   0.28   & 1.02215(5)  & 1.02164(1)   & -		&0.99263 \{19\} 	& 1.0217 \{23\}  & 1.0216441 \{19\}\\
          &   0.5    & 1.66021(5)  & 1.65977(1)   & 1.65975(2)&1.643871 \{19\}	& 1.6599 \{23\}  & 1.6597723 \{19\}\\
          &   1.0    & 3.00030(5)  & 3.00000(1)   & 3.00000(3)&2.9902683 \{19\}	& 3.0002 \{23\}  & 3.0000001 \{19\}\\
\cline{2-8}
\multicolumn{8}{c}{} \\
    6     &   0.1    &  3.5690(3)  &  3.55385(5)  & -		&3.49991 \{18\} 	& 3.5805 \{22\}  & 3.551776 \{9\}\\
          &   0.28   &  7.6216(4)  &  7.60019(6)  & 7.6001(1) &7.56972 \{18\} 	& 7.6254 \{22\}  & 7.599579 \{6\}\\
          &   0.5    & 11.8103(4)  & 11.78484(6)  & 11.7888(2)&11.76228 \{18\}	& 11.8055 \{22\} & 11.785915 \{6\}\\
          &   1.0    & 20.1902(4)  & 20.15932(8)  & 20.1597(2)&20.14393 \{18\}	& 20.1734 \{22\} & 20.160472 \{8\}\\
\cline{2-8}
\multicolumn{8}{c}{} \\
    12    &   0.1    & 12.3162(5)  & 12.26984(8)  & - 		&12.2253 \{17\} 	& 12.3497 \{21\} & 12.850344 \{3\}\\
          &   0.28   & 25.7015(6)  & 25.63577(9)  & - 		&25.61084 \{17\} 	& 25.7095 \{21\} & 26.482570 \{2\}\\
          &   0.5    & 39.2343(6)  & 39.1596(1)   & 39.159(1) &39.13899 \{17\}	& 39.2194 \{21\} & 39.922693 \{2\}\\
          &   1.0    & 65.7905(7)  & 65.7001(1)   & 65.700(1) &65.68304 \{17\}	& 65.7399 \{21\} & 66.076116 \{3\}\\
\cline{2-8}
\multicolumn{8}{c}{} \\
    20    &   0.1    &  30.0729(8)  &  29.9779(1) & -		&29.95345 \{16\}	& 30.2700 \{8\} & 34.204867 \{1\}\\
          &   0.28   &  62.0543(8)  &  61.9268(1) & 61.922(2) &61.91368 \{16\}	& 62.0676 \{20\} & 67.767987 \{1\}\\
          &   0.5    &  94.0236(9)  &  93.8752(1) & 93.867(3) &93.86145 \{16\}	& 93.9889 \{20\} & 100.93607 \{1\}\\
          &   1.0    & 156.062(1)   & 155.8822(1) & 155.868(6)&155.8665 \{16\}	& 155.9569 \{20\}& 164.61280 \{1\}\\
\cline{2-8}
\multicolumn{8}{c}{} \\
    30    &   0.1    &  60.584(1)  &  60.4205(2)  & -		&60.43000 \{15\}	&  61.3827 \{9\}& -\\
          &   0.28   & 124.181(1)  & 123.9683(2)  & - 		&123.9733 \{15\}	& 124.2111 \{9\}& -\\
          &   0.5    & 187.294(1)  & 187.0426(2)  & - 		&187.0408 \{15\}	& 187.2231 \{19\}& -\\
          &   1.0    & 308.858(1)  & 308.5627(2)  & -	 	&308.5536 \{15\}	& 308.6810 \{19\}& -\\
\cline{2-8}
\multicolumn{8}{c}{} \\
    42    &   0.1    & 107.881(1)  & 107.6389(2)  & - 		&- 			& 111.7170 \{8\}& -\\
          &   0.28   & 220.161(1)  & 219.8426(2)  & - 		&219.8836 \{14\}	& 222.1401 \{8\}& -\\
          &   0.5    & 331.002(1)  & 330.6306(2)  & - 		&330.6485 \{14\}	& 331.8901 \{8\}& -\\
          &   1.0    & 544.2(8)    & 542.9428(8)  & - 		&542.9528 \{14\}	& 543.1155 \{18\}& -\\
\cline{2-8}
\multicolumn{8}{c}{} \\
    56    &   0.1    & 176.269(2) & 175.9553(7)   & -		& -		& 186.1034 \{9\} & -		\\
          &   0.28   & 358.594(2) & 358.145(2)    & -		& -		& 363.2048 \{9\} & -		\\
          &   0.5    & 538.5(6)   & 537.353(2)    & -		& -		& 540.3430 \{9\} & -		\\
          &   1      & 880.2(7)   & 879.3986(6)   & -		& -		& 879.6386 \{17\}& -		\\
\hline\hline


\end{tabular}
\caption{Ground state energy results for two-dimensional $N$-electron quantum dots with frequency $\omega$. Refs. $(a)$: F. Pederiva \cite{MagnusArticle} (DMC), $(b)$: S. Reimann \cite{Sarah} (Similarity Renormalization Group theory), $(c)$: C. Hirth \cite{Hirth} (Coupled Cluster Singles and Doubles), $(d)$: V. K. B. Olsen \cite{Olsen} (Full Configuration Interaction). The numbers inside curly brackets denote the number of shells used above the last filled shell, i.e.~above the so-called \textit{Fermi-level} \cite{Shavitt}, to construct the basis for the corresponding methods.}
\label{tab:QDotsResultsAll}
\end{center}
\end{table}
\setlength{\tabcolsep}{6pt}


\clearpage
\subsubsection{Three-dimensional quantum dots}

\setlength{\tabcolsep}{1.05cm}
\begin{table}
\begin{center}
\begin{tabular}{cc|rrr}
    N     & $\omega$ & $\mathrm{E_{VMC}}$ & $\mathrm{E_{DMC}}$ & $E_\mathrm{ref}$\\
\hline\hline
\multicolumn{5}{c}{} \\
    2     &   0.01   & 0.07939(3)  & 0.079206(3) & -		\\
          &   0.1    & 0.50024(8)  & 0.499997(3) & 0.5        \\
          &   0.28   & 1.20173(5)  & 1.201725(2) & -		\\
          &   0.5    & 2.00005(2)  & 2.000000(2) & 2.0 \\
          &   1.0    & 3.73032(8)  & 3.730123(3) & - \\
\cline{2-5}
\multicolumn{5}{c}{} \\
    8     &   0.1    & 5.7130(6)   & 5.7028(1)   & - 		\\
          &   0.28   & 12.2040(8)  & 12.1927(1)  & -		\\
          &   0.5    & 18.9750(7)  & 18.9611(1)  & -\\
          &   1.0    & 32.6842(8)  & 32.6680(1)  & -\\
\cline{2-5}
\multicolumn{5}{c}{} \\
    20    &   0.1    & 27.316(2)   & 27.2717(2)   & - 		\\
          &   0.28   & 56.440(2)   & 56.3868(2)   & -		\\
          &   0.5    & 85.714(2)   & 85.6555(2)   & - \\
          &   1.0    & 142.951(2)  & 142.8875(2)  & -\\
\cline{2-5}
\multicolumn{5}{c}{} \\
\hline\hline
\end{tabular}
\caption{Ground state energy results for three-dimensional $N$-electron quantum dots with frequency $\omega$. The values in the fifth column is exact calculations taken from Ref. \cite{taut}. The VMC result is as expected always higher than the corresponding DMC result.}
\label{tab:QDotsResults3D}
\end{center}
\end{table}
\setlength{\tabcolsep}{6pt}

The results for three-dimensional quantum dots are presented in Table \ref{tab:QDotsResults3D}. Three-dimensional quantum dots do not have the same foothold in literature as the two-dimensional ones, hence no results are listed except for some exact solutions taken from Ref. \cite{taut}. 

As expected, DMC reproduces the exact results for two particles. Compared to the exact results for two dimensions, which was reproduced with five digit precision, the exact results are reproduced with six decimal precision for three dimensions. This strongly indicates that for two electrons, DMC behaves better for three dimensions than for two. For higher number of particles, however, the errors are of the same order of magnitude as for two dimensions, leading to the conclusion that DMC performs equally good in either dimension for quantum dots. 

\subsection{One-body Densities}

The one-body densities are calculated using the methods described in Section \ref{sec:OBD}.

Figure \ref{fig:OBD_DMC_QDOTS_w1} presents the one-body densities for two-dimensional quantum dots. It is clear that the distributions are following a clear trend: The densities in the left column, that is, the densities for $N=2$, 12, and 30 particles, are all similar in shape. The shape for the $N=2$ case can be seen as the top of the $N=12$ density, which in turn can be seen as the top of the $N=30$ density. A physical explanation to this is that the shapes are conserved due to the fact that they represent energetically favorable configurations. 

The same trend is present for the distributions in the right column, that is, the densities for $N=6$, 20, and 42 particles. Viewing the distributions as a sequence of images, from the lowest number of particles to the highest, it is apparent that the shape propagates very much like water ripples. It is remarkable how the solutions to the most complex of equations can come in the form of simple patterns found all around nature.


\clearpage
\captionsetup[subfloat]{labelformat=empty}
\begin{figure}
 \begin{center}
  \subfigure[$N=2$]{\includegraphics[scale=0.35]{../Graphics/OBD/OBD_DMC/dist_out_QDots2c1_3D.png}}
  \subfigure[$N=6$]{\includegraphics[scale=0.35]{../Graphics/OBD/OBD_DMC/dist_out_QDots6c1_3D.png}} \\
  \subfigure[$N=12$]{\includegraphics[scale=0.35]{../Graphics/OBD/OBD_DMC/dist_out_QDots12c1_3D.png}}
  \subfigure[$N=20$]{\includegraphics[scale=0.35]{../Graphics/OBD/OBD_DMC/dist_out_QDots20c1_3D.png}} \\
  \subfigure[$N=30$]{\includegraphics[scale=0.35]{../Graphics/OBD/OBD_DMC/dist_out_QDots30c1_3D.png}}
  \subfigure[$N=42$]{\includegraphics[scale=0.35]{../Graphics/OBD/OBD_DMC/dist_out_QDots42c1_3D.png}} \\
  \caption{Diffusion Monte-Carlo one-body densities for two-dimensional quantum dots with frequency $\omega=1$. The number of particles $N$ are listed below each density. It is apparent that the density behaves much like water ripples as the number of particles increase, conserving the shape in an oscillatory manner.}
  \label{fig:OBD_DMC_QDOTS_w1} 
 \end{center}
\end{figure}

\clearpage


Due to the electron-electron interaction, the Schrödinger equation is not separable in Cartesian coordinates. It is therefore not given that the insights from two dimensions can be transferred to the three-dimensional case. Nevertheless, by looking at the one-body densities for three dimensions in Figure \ref{fig:OBD_QDOTS3D_highfreq}, it is apparent that the general density profile is unchanged. The only thing separating two - and three-dimensional quantum dots is the number of electrons in the closed shells.

Note however, that this similarity only holds when the number of closed shells are equal. Comparing the two-dimensional density for $N=20$ electrons from Figure \ref{fig:OBD_DMC_QDOTS_w1} with the three-dimensional one for $N=20$ electrons given above, it is apparent that the shape of the densities are not conserved with respect to the number of particles $N$ alone.

% \setlength{\tabcolsep}{0.1pt}
\begin{figure}
 \begin{center}
 \begin{tabular}{cc|c}
   \subfigure{\includegraphics[scale=0.3]{../Graphics/OBD/OBD_Q3D/QD2w1_3D.png}} &
   \subfigure{\includegraphics[scale=0.25]{../Graphics/OBD/OBD_Q3D/QD2w1_2D.png}} &
   \subfigure{\includegraphics[scale=0.25]{../Graphics/OBD/OBD_Q3D/comp/Q2D_2.png}} \\
   \subfigure{\includegraphics[scale=0.3]{../Graphics/OBD/OBD_Q3D/QD8w1_3D.png}} &
   \subfigure{\includegraphics[scale=0.25]{../Graphics/OBD/OBD_Q3D/QD8w1_2D.png}} & 
   \subfigure{\includegraphics[scale=0.25]{../Graphics/OBD/OBD_Q3D/comp/Q2D_6.png}} \\
   \subfigure{\includegraphics[scale=0.3]{../Graphics/OBD/OBD_Q3D/QD20w1_3D.png}} &
   \subfigure{\includegraphics[scale=0.25]{../Graphics/OBD/OBD_Q3D/QD20w1_2D.png}} & 
   \subfigure{\includegraphics[scale=0.25]{../Graphics/OBD/OBD_Q3D/comp/Q2D_12.png}} \\
  \end{tabular}
  \caption{Left and middle column: One-body densities for quantum dots in three dimensions with frequency $\omega=1$. A quarter of the spherical density is removed to present a better view of the core.  Red and blue color indicate a low and high electron density, respectively. From top to bottom, the number of particles are $2$, $8$ and $20$. Right column: One-body densities for two-dimensional quantum dots for $N=2$, $6$ and $12$ electrons (from the top) with $\omega=1$. It is apparent that the  shape of the density is conserved as the third dimension is added. The radial densities are not normalized. Normalizing the densities would only change the vertical extent.}
  \label{fig:OBD_QDOTS3D_highfreq}
 \end{center}
\end{figure}
% \setlength{\tabcolsep}{pt}

\captionsetup[subfloat]{labelformat=parens}

\clearpage

\subsection{Lowering the frequency}

An interesting effect of lowering the frequency is that the two - and three-dimensional densities no longer match. For example, the radial density for the three-dimensional $N=8$ quantum dot from Figure \ref{fig:OBD_QDOTS3D_highfreq} was a near perfect match to the two-dimensional one for $N=6$ electrons, however, comparing the same densities for $\omega=0.01$ from Figures \ref{fig:OBD_QDOTS3D_lowfreq} and \ref{fig:OBD_DMC_QDOTS_lowering3D}, it is apparent that this is no longer the case; the two-dimensional density has a peak in the center region, whereas the three-dimensional density is zero in the center region.

From Figure \ref{fig:OBD_QDOTS3D_lowfreq} it is apparent that lowering the frequency increases the radial extent of the quantum dot, and thus lowers the electron density. Moreover, Figure \ref{fig:OBD_DMC_QDOTS_lowering3D} reveals that the electron density becomes similar in height and more localized across the quantum dot, which implies that the electrons on average are spread evenly in shell structures. The localization of the electrons is further verified in Figure \ref{fig:E_dist_qdots}, where is is clear that the expectation value of the total potential energy becomes larger than the corresponding kinetic energy.

In order words, an evenly spread and localized electron density gives rise to \textit{crystallization}\footnote{Unless at least one particle is frozen in the QMC simulations, the quantum dot densities should always be rotationally symmetric. Crystallization in a QMC perspective comes thus not in the form of actual ``crystals'', but rather as a rotated crystallized state.}. The idea of an electron crystal was originally proposed by Wigner \cite{WignerCrystalOrig}, hence the currently discussed phenomenon is referred to as a \textit{Wigner molecule} or a \textit{Wigner crystal}, which is expected for quantum dots in the limit of low electron densities where the total potential energy dominates over the kinetic energy \cite{WignerTransport, WignerPathTo, WignerSymmetryBreak, WignerFloating, Wigner2DQD}. These electronic crystals have been observed in experiments with for example liquid Helium \cite{WignerExptHelium} and semiconductors \cite{WignerExptSemicond}.

\newcommand{\qqq}{\qquad\qquad\qquad}
\newcommand{\qq}{\qquad\qquad}
\newcommand{\rot}[1]{\begin{sideways}#1\end{sideways}}
\setlength{\tabcolsep}{0.1pt}
\begin{figure}
 \begin{center}
 \begin{tabular}{rl}
   \rot{$\qq\quad\omega=1$}&\subfigure{\includegraphics[scale=0.35]{../Graphics/OBD/OBD_Q3D/QD8w1_3D.png}}
   \subfigure{\includegraphics[scale=0.28]{../Graphics/OBD/OBD_Q3D/QD8w1_2D.png}} \\
   \rot{$\qq\omega=0.01$} &\subfigure{\includegraphics[scale=0.35]{../Graphics/OBD/OBD_Q3D/QD8w001_3D.png}} 
   \subfigure{\includegraphics[scale=0.28]{../Graphics/OBD/OBD_Q3D/QD8w001_2D.png}}  \\
  \end{tabular}
  \caption{Comparison of the one-body densities for quantum dots in three dimensions for $N=8$ electrons for high and low frequency $\omega$. It is apparent that the distribution becomes more narrow as the frequency is reduced. Red and blue color indicate a low and high electron density, respectively. A quarter of the spherical density is removed to present a better view of the core. }
  \label{fig:OBD_QDOTS3D_lowfreq}
 \end{center}
\end{figure}

\captionsetup[subfloat]{labelformat=empty}
\newcommand{\OBDscale}{0.25}
\begin{landscape}
 \begin{figure}
 \begin{center}
 \begin{tabular}{rl}
  \rot{$\qquad\quad\omega=0.28$}&\subfigure{\includegraphics[scale=\OBDscale]{../Graphics/OBD/OBD_DMC/dist_out_QDots2c028_3D.png}}
  \subfigure{\includegraphics[scale=\OBDscale]{../Graphics/OBD/OBD_DMC/dist_out_QDots6c028_3D.png}} 
  \subfigure{\includegraphics[scale=\OBDscale]{../Graphics/OBD/OBD_DMC/dist_out_QDots12c028_3D.png}}
  \subfigure{\includegraphics[scale=\OBDscale]{../Graphics/OBD/OBD_DMC/dist_out_QDots20c028_3D.png}} \\[-0pt]
  \rot{$\qquad\quad\omega=0.1$}&\subfigure{\includegraphics[scale=\OBDscale]{../Graphics/OBD/OBD_DMC/dist_out_QDots2c01_3D.png}}
  \subfigure{\includegraphics[scale=\OBDscale]{../Graphics/OBD/OBD_DMC/dist_out_QDots6c01_3D.png}} 
  \subfigure{\includegraphics[scale=\OBDscale]{../Graphics/OBD/OBD_DMC/dist_out_QDots12c01_3D.png}}
  \subfigure{\includegraphics[scale=\OBDscale]{../Graphics/OBD/OBD_DMC/dist_out_QDots20c01_3D.png}} \\[-0pt]
  \rot{$\qquad\quad\omega=0.01$}&\subfigure[$N=2$]{\includegraphics[scale=\OBDscale]{../Graphics/OBD/OBD_DMC/dist_out_QDots2c001_3D.png}}
  \subfigure[$N=6$]{\includegraphics[scale=\OBDscale]{../Graphics/OBD/OBD_DMC/dist_out_QDots6c001_3D.png}} 
  \subfigure[$N=12$]{\includegraphics[scale=\OBDscale]{../Graphics/OBD/OBD_DMC/dist_out_QDots12c001_3D.png}}
  \subfigure[$N=20$]{\includegraphics[scale=\OBDscale]{../Graphics/OBD/OBD_DMC/dist_out_QDots20c001_3D.png}} \\
 \end{tabular}
  \caption{\small{DMC One-body densities for Quantum Dots for decreasing oscillator frequencies $\omega$ and increasing number of particles $N$. Each row represents a given $\omega$, and each column represents a given $N$. Notice that the densities for $\omega=1$ (from figure \ref{fig:OBD_DMC_QDOTS_w1}) are indistinguishable from those of $\omega=0.28$ except for their radial extent. This trend has been verified in the case of $N=30$, 42 and 56 electrons as well as for $\omega=0.5$, however, for the sake of transparency, these results are left out of the current figure.}}
  \label{fig:OBD_DMC_QDOTS_lowering3D}
 \end{center}
\end{figure}
\end{landscape}

\captionsetup[subfloat]{labelformat=parens}
\setlength{\tabcolsep}{6pt}


\captionsetup[subfloat]{labelformat=empty}
\begin{figure}[h]
 \begin{center}
  \subfigure[$N=6$]{\includegraphics[scale=0.35]{../Graphics/VirialPlots/E_vs_w_E6.png}}
  \subfigure[$N=42$]{\includegraphics[scale=0.35]{../Graphics/VirialPlots/E_vs_w_E42.png}} \\
  \caption{The relative magnitude of the expectation value of the different energy sources as a function of the frequency $\omega$ (left) together with the magnitude of the sources' energy contributions scaled with the oscillator frequency (right). The plots are supplied with legends to increase the readability. The different energy sources are the kinetic energy denoted \textit{Ekin}, the oscillator potential energy denonted \textit{Eosc}, and the electron-electron interaction energy denoted \textit{Ecol}. Note that all given energies are expectation values. The values are calculated using two-dimensional quantum dots. The number of electrons $N$ is displayed beneath each respective plot. It is apparent that the kinetic energy contribution is constant in both cases. Moreover, the oscillator potential contribution is more or less constant for the relative energies (left sub-figures). The figure clearly indicates that the potential energy contributions from the oscillator and the electron-electron interaction tends to dominate over the kinetic energy at lower frequencies.}
  \label{fig:E_dist_qdots}
 \end{center}
\end{figure}

It is expected that the QMC Wigner crystal corresponds to the electrons localizing around the equilibrium positions of the classical Wigner crystal\cite{WignerTransport}. Comparing the densities for two-dimensional quantum dots at $\omega=0.01$ for $N=6$, $12$, and $20$ electrons given in Figure \ref{fig:OBD_DMC_QDOTS_lowering3D} to similar classical calculations done in Ref. \cite{WignerClassic} for $N=6$, $10$, and $19$ electrons, respectively, it is apparent that the solutions match very well.  

It was mentioned previously that the Wigner crystallization of quantum dots came as a consequence of the average total potential energy being larger than the corresponding kinetic energy. This relationship between kinetic - and potential energy is closely related to the \textit{virial theorem} from classical mechanics. The quantum mechanical version of the virial theorem was proven by Fock in 1930 \cite{FockVirial} and reads

\begin{equation}
 V(\mathbf{r}) \propto r^\gamma \quad\longrightarrow\quad \langle \OP{T} \rangle = \frac{\gamma}{2} \langle \OP{V} \rangle, \label{eq:virial}
\end{equation}

where $\OP{T}$ and $\OP{V}$ denote the kinetic - and total potential energy operators, respectively. The important conclusion which can be drawn from this is that if two systems have an equal ratio of kinetic - to total potential energy, the systems behave identically in the sense that they follow the same effective potential, and thus have similar eigenstates. 

From Figure \ref{fig:V_dist_qdots} it is apparent that there is a remarkably constant slope for two different regions in the case of quantum dots, namely high - and low kinetic energy, which by looking at Figure \ref{fig:E_dist_qdots} corresponds to high - and low frequencies. In light of previous discussions, one may suggest that the change in the slopes
of Figure \ref{fig:V_dist_qdots} corresponds to the quantum dot system  making a transition into a Wigner crystallized state.



\newpage

\begin{figure}[h]
 \begin{center}
  \subfigure[$N=6$]{\includegraphics[scale=0.35]{../Graphics/VirialPlots/E_vs_w_V6.png}}
  \subfigure[$N=42$]{\includegraphics[scale=0.35]{../Graphics/VirialPlots/E_vs_w_V42.png}} \\
  \caption{The total kinetic energy vs. the total potential energy of two-dimensional quantum dots. The number of electrons $N$ are displayed beneath each respective plot. The axes are scaled with a power of $N$ to collapse the data to the same axis span. Once the kinetic energy drops below a certain energy dependent on the number of particles, the slope changes, which in light of the virial theorem from Eq.~(\ref{eq:virial}) indicates that the overall system changes properties. The data is fitted to linear lines with resulting slopes $a$ displayed in the legend. The parameter $r2$ indicates how well the data fits a linear line. An exact fit yields $r2 = 1$.}
  \label{fig:V_dist_qdots}
 \end{center}
\end{figure}
\captionsetup[subfloat]{labelformat=parens}

\subsection{Simulating a Double-well}

\begin{figure}[h]
 \begin{center}
 \includegraphics[scale=0.35]{../Graphics/DoubleWell.png}
  \caption{A countour plot of the trial wave function for a two-particle double-well quantum dot with the wells separated at a distance $R = 2$ in the $x$-direction using $m^\ast = \omega_0 = 1$. See Section \ref{sec:modelQDots} for an introduction to the double well potential. It is apparent that there is one electron located in each well, however, with a slight overlap in the middle region.}
  \label{fig:doubleWell}
 \end{center}
\end{figure}
\captionsetup[subfloat]{labelformat=parens}

Figure \ref{fig:doubleWell} shows the distribution for a two-particle simulation. It is apparent that despite the wells being separated, their local distributions overlap, indicating that the electrons can \textit{tunnel}, that is, they have a non-zero probability of appearing on the other side of the barrier. Comparing the distribution to the potential in Figure \ref{fig:extPotDoubleWell}, it is clear that they match very well.

The DMC result is $\mathrm{E_{DMC}} = 2.3496(1)$, where as the non-interacting energy is $E_0 = 2$ and the single-well energy is $E_0=3.0$. It is expected that the energy lands in between these, as $R=0$ corresponds to the single well and $R\to\infty$ corresponds to the non-interacting case. 
