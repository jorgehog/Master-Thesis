\chapter{Generalization and Optimization}
\label{ch:optAndGen}

There is a big difference in strategy between writing code for a specific problem, and creating a general solver. A general Quantum Monte-Carlo (QMC) solver involves several layers of complexity, such as support for different potentials, single particle bases, sampling models, etc., which may easily lead to a combinatoric disaster if the planning is not done right. 

This chapter begins by introducing a list of underlying assumptions regarding the modelled systems. Whether or not a system can be solved is then a question of whether the listed assumptions are valid for the system or not. The next part will cover generalization, that is, the flexibility of the code and the strategies used to obtain this flexibility. The result of a generalized code is that different systems and algorithms can be implemented by making simple changes to the code. Finally, optimizations will be covered. Optimizations are crucial in order to maintain efficiency for a high number of particles. 

\section{Underlying Assumptions and Goals}
\label{sec:AssGoal}

In large computational projects it is custom to plan every single part of the program before the actual process of coding begins. Coding massive frameworks without planning almost exclusively result in unforeseen consequences, rendering the code difficult to expand, disorganized, and inefficient. The code used in this thesis has been completely restructured four times. This section will cover the assumptions regarding the modelled systems and the goals regarding generalization and optimization made in the planning stages preliminary to the coding process.

\subsection{Assumptions}
\label{sec:ass}

The code structure was designed based on the following assumptions

\cfbox{3cm-3pt}{
\begin{enumerate}[label=\textbf{(\roman{*})}, ref=(\roman{*}), align=left]
 \item The particles of the simulated systems are either all fermions or all bosons. \label{it::ass::pureFermBos}
 \item The Hamiltonian is spin - and time independent.\label{it::ass::2level}
 \item The trial wave function of a fermionic system is a single determinant. \label{it::ass::fermiSingleDet}
 \item A bosonic system is modelled by all particles being in the same assumed single particle ground state. \label{it::ass::bosCondensate}
 \vspace{0.3cm}
\end{enumerate}
}

As discussed in Section \ref{sec:trialWF}, the second assumption implies that the Slater determinant can be split into parts corresponding to different spin eigenvalues. The time-independence is a requirement on the QMC solver explained in Section \ref{sec:statingDiff}. The assumptions listed are considered true for any system which is implemented in the code, and will thus be applied in all of the following sections. 

\subsection{Generalization Goals}
\label{sec:genGoals}

The implementation should be general for:

\cfbox{3.17cm-1pt}{
\begin{enumerate}[label=\textbf{(\roman{*})}, ref=(\roman{*}), align=left]
 \item Fermions and bosons. \label{it::gen::FermiAndBoson}
 \item Anisotropic- and isotropic diffusion, i.e.~Brute Force - or Importance sampling. \label{it::gen::BF_IS}
 \item Different gradient descent algorithms. 
 \item Any Jastrow factor.
 \item Any error estimation algorithm.
 \item Any single particle basis, including expanded single particle bases. \label{it::gen::SP_basis}
 \item Any combination of any potentials. \label{it::gen::pot}
 \vspace{0.2cm}
\end{enumerate}
}
 
In addition, the following constraint is set on solvers:

\cfbox{3.42cm-1pt}{
\begin{enumerate}[label=\textbf{(\roman{*})}, ref=(\roman{*}), align=left]
\setcounter{enumi}{7}
 \item Full numerical support for all values involving derivatives.\label{it::gen::numSupp}
 \vspace{0.2cm}
\end{enumerate}
}

The challenge is, despite the vast array of combinations, to preserve simplicity and structure as layers of complexity are added. Achieving generalization by the use of conditional if tests inside the solvers is considered inefficient, and should only be used if no other solution is apparent or exists.

\subsection{Optimization Goals}
\label{sec:optGoals}

Modern computers have a vast amount of physical memory available, which makes runtime optimizations favored over memory optimizations. The following list may appear short, but every step brings immense amounts of complexity to the implementation

\cfbox{3.6cm}{
\begin{enumerate}[label=\textbf{(\roman{*})}, ref=(\roman{*}), align=left]
 \item Identical values should never be re-calculated. \label{it::opt::reCalc}
 \item Generalization should not be achieved through conditional if tests in repeated function calls, but rather through polymorphism (see Section \ref{sec:typeCastPoly}). \label{it::opt::noIF}
 \item Linear scaling of runtime vs. the number of processors (CPUs) for large simulations. \label{it::opt::parScale}
 \vspace{0.2cm}
\end{enumerate}
}

Goal (\ref{it::opt::noIF}) has been the ground pillar of the code design. 

\clearpage
\section{Specifics Regarding Generalization}
\label{sec:specGen}

This section will introduce how object orientation is used to achieve the goals regarding generalization of the code. Several examples are discussed, however, for more details regarding the implementation of methods, see the code in Ref. \cite{libBorealisCode}.

\subsection{Generalization Goals \ref{it::gen::FermiAndBoson}-\ref{it::gen::pot}}

As discussed in Section \ref{sec:manyBodyWFs}, the mathematical difference between fermions and bosons (of importance to QMC) is how the many-body wave functions are constructed from the single-particle basis. In the case of fermions, the expression is given in terms of a Slater determinant which, due to the fact that the Hamiltonian is spin-independent, can be split into two parts corresponding to spin up and spin down. On the other hand, for bosons, it is simply the product of all states due to the fact that they are all assumed to occupy the same orbital. This is demonstrated in the following code example, where the wave functions of fermionic and bosonic systems are evaluated:

\vspace{0.2cm}
\begin{lstlisting}[caption=The implementation of the evaluation of fermionic and bosonic wave functions. Line 5: The fermion class accesses the walker's single particle state matrix and returns the determinant of the first half (spin up) times the determinant of the second half (spin down). Line 14-16: The boson class simply calculates the product of all the single particle states.]
double Fermions::get_spatial_wf(const Walker* walker) {
    using namespace arma;
    
    //Spin up times Spin down (determinants)
    return det(walker->phi(span(0, n2 - 1), span())) * det(walker->phi(span(n2, n_p - 1), span()));
}

double Bosons::get_spatial_wf(const Walker* walker) {
 
    double wf = 1;
 
    //Using the phi matrix as a vector in the case of bosons.
    //Assuming all particles to occupy the same single particle state (neglecting permutations).
    for (int i = 0; i < n_p; i++){
      wf *= walker->phi(i);
    }
    
    return wf;
}
\end{lstlisting}

Overloaded pure virtual methods for fermions and bosons exist for all of the methods which involves evaluating the many-body wave function, for example the spatial ratio and the sum of Laplacians. When the \verb+QMC+ solver asks the \verb+System*+ object for a spatial ratio, depending on whether fermions or bosons are loaded run-time, the fermion or boson spatial ratio is evaluated. 

It is apparent that this way of implementing the system class takes care of optimization goal \ref{it::opt::noIF} in Section \ref{sec:optGoals} regarding no use of conditional if tests to determine the nature of the system. 

A similar polymorphic splitting is introduced in the following classes:

\begin{listliketab}
\storestyleof{itemize}
 \begin{tabular}{l l}
 \textbullet \,\verb+Orbitals+       & The hydrogen-like or the harmonic oscillator orbitals. \\
 \textbullet \,\verb+BasisFunctions+ & Stand-alone single particle wave functions initialized by \verb+Orbitals+. \\
 \textbullet \,\verb+Sampling+       & Brute force - or importance sampling. \\
 \textbullet \,\verb+Diffusion+      & Isotropic or Fokker-Planck diffusion. Automatically selected by \verb+Sampling+. \\
 \textbullet \,\verb+ErrorEstimator+ & Simple or Blocking. \\
 \textbullet \,\verb+Jastrow+        & Padé Jastrow - or no Jastrow factor. \\
 \textbullet \,\verb+QMC+            & Variational - (VMC) or diffusion Monte-Carlo (DMC). \\
 \textbullet \,\verb+Minimizer+      & Adaptive stochastic gradient descent (ASGD). \\
 \end{tabular}
\end{listliketab}

Implementing, for example, a new Jastrow Factor is done by simply creating a new subclass of \verb+Jastrow+. The QMC solver does not need to change to adapt to the new implementation. For more details, see Section \ref{sec:OO}. The splitting done in \verb+QMC+ is done to avoid rewriting a lot of general QMC code, such as diffusing walkers.

A detailed description of the generalization of potentials, i.e.~generalization goal \ref{it::gen::pot}, is given in Section \ref{sec:typeCastPoly}.

\subsection{Generalization Goal \ref{it::gen::SP_basis} and Expanded bases}

An expanded single particle basis is implemented as a subclass of the \verb+Orbitals+ superclass. It wraps around an \verb+Orbitals+ implementation, e.g.~the harmonic oscillator orbitals, containing basis elements $\phi_\alpha(\mathbf{r}_j)$. In addition to these elements, the expanded basis class has a set of expansion coefficients $\mathbf{C}_{\gamma\alpha}$ from which the new basis elements are constructed

\begin{equation}
\label{eq:ExpBasisSP}
 \psi_\gamma^\mathrm{Exp.}(r_j) = \sum_{\alpha=0}^{B - 1} \mathbf{C}_{\gamma\alpha}\phi_\alpha(r_j), 
\end{equation}

where $B$ is the size of the expanded basis. The following code snippet presents the vital members of the expanded basis class: 

\vspace{0.5cm}
\begin{lstlisting}[language=C++, caption={The declaration of the expanded basis class. The vital members are the size of the basis, the expansion coefficients and another basis in which the new are expanded. A method for calculating the coefficients is present, but the actual implementation has not been a focus of this thesis.}]
class ExpandedBasis : public Orbitals {

...

protected:

    int basis_size;
    arma::mat coeffs;
    Orbitals* basis;
    
    void calculate_coefficients();

};

\end{lstlisting}


The implementation of Eq.~(\ref{eq:ExpBasisSP}) into the expanded basis class is achieved by overloading the original \verb+Orbitals::phi+ virtual member function as shown in the following example

\vspace{0.5cm}
\begin{lstlisting}[caption=The explicit implementation of the expanded basis single particle wave function. The wave function is evaluated by expanding a given basis in a set of expansion coefficients (see the previous code example).]
double ExpandedBasis::phi(const Walker* walker, int particle, int q_num) {

    double value = 0;
    
    //Dividing basis_size by half assuming a two-level system.
    for (int m = 0; m < basis_size/2; m++) {
        value += coeffs(q_num, m) * basis->phi(walker, particle, m);
    }

    return value;

}
\end{lstlisting}

Expanded bases has not been a focus for the thesis, thus explicit algorithms for calculating the coefficients will not be covered. The reason the implementation has been presented, is to lay the foundation in case future Master students are to expand upon the code.

\subsection{Generalization Goal \ref{it::gen::numSupp}}

Support for evaluating derivatives numerically is important for two reasons; the first being debugging, the second being the cases where no closed-form expressions for the derivatives can be obtained or become too expensive to evaluate.

As an example, the orbital class implementation responsible for the gradient of the single particle states, \verb+Orbitals::del_phi+, is virtual. This implies that it can be overloaded to call the numerical derivative implementation \verb+Orbitals::num_diff+. The same goes for the Laplacian, the Jastrow factor derivatives, and the variational derivatives in the minimizer. An alternative to numerically evaluating the derivatives of the single particle wave functions would be to perform the derivative on the full many-body wave function, however, this would not fit naturally into the code design.

The implemented numerical derivatives are finite difference schemes with an error proportional to the square of the chosen step length.

\section{Optimizations due to a Single two-level Determinant}
\label{sec:optSingleSlater}

Assuming the trial wave function to consist of a single term unlocks several optimizations involving the Slater-determinant. Similar optimizations for bosons are considered trivial in comparison and will not be covered in detail. See the code in \cite{libBorealisCode} for details regarding bosons.

Writing the Slater determinant as the determinant of the \textit{Slater matrix} $\mathbf{S}$, the expression for the trial wave function in Eq.~(\ref{eq:singleDeterminantTWF}) becomes

\newcommand{\PTd}{|\mathbf{S}^\uparrow||\mathbf{S}^\downarrow |J}
\newcommand{\Du}{|\mathbf{S}^\uparrow|}
\newcommand{\Dd}{|\mathbf{S}^\downarrow|}
\newcommand{\Da }{|\mathbf{S}^{\alpha}|}
\newcommand{\Daa}{|\mathbf{S}^{\overline{\alpha}}|}
\newcommand{\PTda}{\Da\Daa J}

\begin{equation}
 \Psi_T = \PTd, \label{eq:phiTrialOptGen}
\end{equation}

where the splitting of the Slater determinant described in parts corresponding to spin up and spin down has been applied (see Eq.~(\ref{eq:splitSlater})). The Jastrow-factor, $J$, is described in section \ref{sec:ChoiceTrialWF}. Function arguments are skipped to clean up the expression. 

Several quantities involves evaluating the trial wave function, one of which are the quantum force from section \ref{sec:anisFokker} responsible for pushing the walkers in the direction of higher probabilities. Written in terms of Eq.~(\ref{eq:phiTrialOptGen}), the expression for the quantum force of particle $i$ becomes 

\begin{eqnarray}
 \mathbf{F}_i &=& 2\frac{\nabla_i\left(\PTd\right)}{\PTd} \nonumber\\
     &=& 2\left(\frac{\nabla_i\Du}{\Du} + \frac{\nabla_i\Dd}{\Dd} + \frac{\nabla_i J}{J}\right). \nonumber \\
\end{eqnarray}

The important part to realize now, is that particle $i$ \textit{either} has spin up or spin down. This implies that one of the derivatives in the last expression is zero due to the fact that the spins are opposite. Denoting the spin of particle $i$ as $\alpha$ and the opposite spin as $\overline{\alpha}$, the expressions for the quantum force reads

\begin{align}
 \mathbf{F}_i &= 2\left(\frac{\nabla_i\Da}{\Da} + \frac{\nabla_i\Daa}{\Daa} + \frac{\nabla_i J}{J}\right), \\ \nonumber
\end{align}

where 

\begin{equation}
 \nabla_i\Daa = 0, \label{eq:differentSpinDie}
\end{equation}

due to the fact that there is no trace of particle $i$ in $\Daa$. The resulting expression involves evaluating the Slater matrix for a single spin configuration only

\begin{equation}
 \mathbf{F}_i = 2\left(\frac{\nabla_i\Da}{\Da} + \frac{\nabla_i J}{J}\right). \nonumber
\end{equation}

The expression for the local energy from Section \ref{sec:calcExpVals} can be simplified in a similar manner. Starting from the original expression

\begin{equation}
 E_L = -\frac{1}{2}\sum_i \frac{1}{\Psi_T}\nabla^2_i \Psi_T + \sum_iV_i, \label{eq:localEsum}
\end{equation}

the Laplacian can be expanded in the same way as was done for the quantum force

 \begin{eqnarray}
  \frac{1}{\Psi_T}\nabla^2_i\Psi_T &=&  \frac{1}{\PTda}\nabla^2_i \PTda \nonumber\\
  &=& \frac{\nabla^2_i \Da}{\Da} + \frac{\nabla^2_i \Daa}{\Daa} + \frac{\nabla^2_i J}{J} \nonumber\\
  && +\,\, 2\frac{\left(\nabla_i\Da\right)\left(\nabla_i\Daa\right)}{\Da\Daa} + 2 \frac{\left(\nabla_i\Da\right)\left(\nabla_i J\right)}{\Da J} + 2 \frac{\left(\nabla_i\Daa\right)\left(\nabla_i J\right)}{\Daa J} \nonumber\\
  &=& \frac{\nabla^2_i \Da}{\Da} + \frac{\nabla^2_i J}{J} + 2\frac{\nabla_i\Da}{\Da} \frac{\nabla_i J}{J},
\end{eqnarray}

where half of the terms vanish due to Eq.~(\ref{eq:differentSpinDie}).

The last expression involving the Slater matrix (which can be simplified) is the spatial ratio, $R_\psi$, used in the Metropolis algorithm (see Section \ref{sec:MetroMain}). The expression reads

\begin{eqnarray}
 R_\psi &=& \frac{\Psi_T^\mathrm{new}}{\Psi_T^\mathrm{old}} \nonumber\\
 &=& \frac{\Du^\mathrm{new} \Dd^\mathrm{new} J^\mathrm{new}}{\Du^\mathrm{old} \Dd^\mathrm{old} J^\mathrm{old}}, \label{eq:R_psi_allspins}\\
\end{eqnarray}

where the old and new superscript denotes the wave function prior to and after moving one particle, respectively. Let again $i$ be the currently moved particle with spin $\alpha$. As discussed previously, the part of the trial wave function representing the opposite spin of $\alpha$, $\Daa$, is independent of particle $i$. This implies that moving particle $i$ does not change the value of $\Daa$, that is

\begin{equation}
 \Daa^\mathrm{new} = \Daa^\mathrm{old}.
\end{equation}

Inserting this into Eq.~(\ref{eq:R_psi_allspins}) in addition to the spin parameters $\alpha$ and $\overline{\alpha}$ gives

\begin{equation}
 R_\psi = \frac{\Daa^\mathrm{new}}{\Daa^\mathrm{old}} = \underbrace{\frac{\Daa^\mathrm{new}}{\Daa^\mathrm{old}}}_{1}\frac{\Da^\mathrm{new}}{\Da^\mathrm{old}}\frac{J^\mathrm{new}}{J^\mathrm{old}} = \frac{\Da^\mathrm{new}}{\Da^\mathrm{old}}\frac{J^\mathrm{new}}{J^\mathrm{old}}. \label{eq:RpsiOpt}
\end{equation}

From the expressions deduced in this section it is clear that the dimensionality of the calculations is halved by splitting the Slater determinant into two parts. Calculating the determinant of a $N\times N$ matrix is $\mathcal{O}(N^2)$ floating point operations (flops), which implies a speedup of four in estimating the determinants alone. 

\section{Optimizations due to Single-particle Moves}

Moving a single particle at the time implies that only a single row in the Slater-determinant from Eq.~(\ref{eq:SlaterDeterminantExplicit}) will be changed between the calculations. Changes to a single row implies that many \textit{co-factors} remain unchanged. Since all of the expressions deduced in the previous section contains ratios of the spatial wave functions, expressing these determinants in terms of their co-factors should reveal a cancellation of terms. 

Expressing the cancellation mathematically presents the possibility to optimize the calculations by implementing only the parts which does not cancel. 

In the previous section it became apparent that only the determinant whose spin level matches that of the moved particle needs to be calculated explicitly. In the following sections, the spin indication on the Slater matrix $\Da$ will be skipped in order to clean up the equations.

\subsection{Optimizing the Slater ratios}
\label{sec:optSlaterRat}

\newcommand{\Sinv}{\mathbf{S}^{-1}}

The inverse of the Slater-matrix introduced in the previous section, is given in terms of its \textit{adjugate} by the following relation \cite{linAlg}

\begin{equation*}
 \Sinv = \frac{1}{|\mathbf{S}|}\mathrm{adj} \mathbf{S}.
\end{equation*}

The adjugate of a matrix is the transpose of the cofactor matrix $\mathbf{C}$. The expression in terms of matrix element equations reads

\begin{eqnarray}
 \Sinv &=& \frac{\mathbf{C}^T}{|\mathbf{S}|}, \\
 \Sinv_{ij} &=& \frac{\mathbf{C}_{ji}}{|\mathbf{S}|}\label{eq:invExpCofac}, \\
 \mathbf{S}_{ji} &=& \phi_i(\mathbf{r}_j). \label{eq:slaterMatPhi}
\end{eqnarray}

Moreover, the determinant can be expressed as a \textit{cofactor expansion} around row $j$ (Kramer's rule) \cite{linAlg}

\begin{equation}
\label{eq:cofacExp}
 |\mathbf{S}| = \sum_i \mathbf{S}_{ji} \mathbf{C}_{ji}.
\end{equation}

The spatial part of the $R_\psi$ ratio is obtained by inserting Eq.~(\ref{eq:cofacExp}) into Eq.~(\ref{eq:RpsiOpt})

\begin{equation}
\label{eq:RpsiCofac}
 R_S = \frac{\sum_i \mathbf{S}_{ji}^\mathrm{new}\mathbf{C}_{ji}^\mathrm{new}}{\sum_i \mathbf{S}_{ji}^\mathrm{old}\mathbf{C}_{ji}^\mathrm{old}}.
\end{equation}

Let $j$ represent the moved particle. The $j$'th column of the cofactor matrix is unchanged when the particle moves (column $j$ depends on every column but its own). In other words

\begin{equation}
 \mathbf{C}_{ji}^\mathrm{new} = \mathbf{C}_{ji}^\mathrm{old} = (\mathbf{S}^\mathrm{old}_{ij})^{-1}|\mathbf{S}^\mathrm{old}|,
\end{equation}

where the inverse relation of Eq.~(\ref{eq:invExpCofac}) has been used. Inserting this into Eq.~(\ref{eq:RpsiCofac}) yields

\begin{eqnarray}
  R_S &=& \frac{|\mathbf{S}^\mathrm{old}|}{|\mathbf{S}^\mathrm{old}|}\frac{\sum_i \mathbf{S}_{ji}^\mathrm{new}(\mathbf{S}_{ij}^\mathrm{old})^{-1}}{\sum_i \mathbf{S}_{ji}^\mathrm{old}(\mathbf{S}_{ij}^\mathrm{old})^{-1}} \nonumber\\
         &=& \frac{\sum_i \mathbf{S}_{ji}^\mathrm{new}(\mathbf{S}_{ij}^\mathrm{old})^{-1}}{\mathbf{I}_{jj}}. \nonumber
\end{eqnarray}

The diagonal element of the identity matrix is by definition unity. Inserting this fact combined with the relation from Eq.~(\ref{eq:slaterMatPhi}), an optimized expression for the ratio is obtained:

\cfbox{4cm-4pt}{
\begin{equation}
\label{eq:spatialRatioDet}
 R_S = \sum_i \phi_i(\mathbf{r}_j^\mathrm{new})(\mathbf{S}_{ij}^\mathrm{old})^{-1},
\end{equation}
}

where $j$ is the currently moved particle. The sum over $i$ spans the Slater matrix whose spin value matches that of particle $j$, i.e either $\mathbf{S}^\alpha$ or $\mathbf{S}^{\overline{\alpha}}$ (see the previous section).

Similar reductions can be applied to all the Slater ratio expressions from the previous section \cite{abInitioMC, morten}:

\cfbox{4cm}{
\begin{eqnarray}
 \frac{\nabla_i|\mathbf{S}|}{|\mathbf{S}|} &=& \sum_{k} \nabla_i\phi_k(r_i^\mathrm{new})(\mathbf{S}_{ki}^\mathrm{new})^{-1}, \\
  \frac{\nabla_i^2|\mathbf{S}|}{|\mathbf{S}|} &=& \sum_{k} \nabla_i^2\phi_k(r_i^\mathrm{new})(\mathbf{S}_{ki}^\mathrm{new})^{-1},
\end{eqnarray}
}

where the sum $k$ spans the Slater matrix whose spin values match that of the moved particle.

Closed form expressions for the derivatives and Laplacians of the single particle wave functions can be implemented in order to avoid expensive numerical calculations. See Appendices \ref{appendix:SymPyHO}, \ref{appendix:SymPyHO3D} and \ref{appendix:SymPyHydro} for a tabulation of closed form expressions used in this Thesis. Appendix \ref{appendix:sympy} presents an efficient strategy for obtaining these expressions.

\subsection{Optimizing the Inverse}
\label{sec:optInv}

One might question the efficiency of calculating inverse matrices compared to brute force estimation of the determinants. The efficiency of the inverse becomes apparent, as with the ratio in Eq.~(\ref{eq:spatialRatioDet}), by co-factor expanding the expression; an updating algorithm which dramatically decreases the cost of calculating the inverse of the new Slater matrix can be implemented.

Let $i$ denote the currently moved particle. The new inverse is given in terms of the previous by the following expression \cite{abInitioMC, morten}

\cfbox{4.5cm-5pt}{
\begin{eqnarray}
 \mathbf{\tilde I}_{ij} &=& \sum_l \mathbf{S}_{il}^\mathrm{new}(\mathbf{S}_{lj}^\mathrm{old})^{-1}, \\
 \ \label{eq:Itilde}\
 (\mathbf{S}_{kj}^\mathrm{new})^{-1} &=& (\mathbf{S}_{kj}^\mathrm{old})^{-1} - \frac{1}{R_S}(\mathbf{S}_{ji}^\mathrm{old})^{-1}\tilde I_{ij} \qquad\qquad j \ne i,\\
 (\mathbf{S}_{ki}^\mathrm{new})^{-1} &=& \frac{1}{R_S}(\mathbf{S}_{ki}^\mathrm{old})^{-1} \qquad\qquad\qquad\qquad\qquad\,\, \mathrm{else}. \\
 \ \nonumber
\end{eqnarray}
}

This reduces the cost of calculating the inverse by an order of magnitude down to $\mathcal{O}(N^2)$.

Further optimization can be achieved by calculating the $\mathbf{\tilde I}$ vector for particle $i$ prior to performing the loop over $k$ and $j$. Again, this loop should only update the inverse Slater matrix whose spin value correspond to that of the moved particle.

\subsection{Optimizing the Padé Jastrow factor Ratio}
\label{sec:optPagejastRat}

Such as was done with the Green's function ratio in Eq.~(\ref{eq:MetropolisHastings}), the ratio between two Jastrow factors are best expressed in terms of the logarithm

\begin{eqnarray}
 \log \frac{J^\mathrm{new}}{J^\mathrm{old}} &=& \sum_{k<j = 1}^N \left[\frac{a_{kj}r^\mathrm{new}_{kj}}{1 + \beta r^\mathrm{new}_{kj}} - \frac{a_{kj}r^\mathrm{old}_{kj}}{1 + \beta r^\mathrm{old}_{kj}}\right] \\
                      &\equiv& \sum_{k<j = 1}^N \left[g^\mathrm{new}_{kj} - g^\mathrm{old}_{kj}\right]. \label{eq:jastrowRatSTD}
\end{eqnarray}

The relative distances $r_{kj}$ behave much like the cofactors in Section \ref{sec:optSlaterRat}: Changing $r_i$ changes only $r_{ij}$, that is

\begin{equation}
 r^\mathrm{new}_{kj} = r^\mathrm{old}_{kj} \qquad k \ne i, 
\end{equation}

which inserted into Eq.~(\ref{eq:jastrowRatSTD}) yields

\begin{eqnarray}
  \log\frac{J^\mathrm{new}}{J^\mathrm{old}} &=& \sum_{k<j \ne i}  \underbrace{\left[g^\mathrm{old}_{kj} - g^\mathrm{old}_{kj}\right]}_{0} + \sum_{j = 1}^N \left[ g^\mathrm{new}_{ij} - g^\mathrm{old}_{ij}\right] \nonumber\\
                                            &=& \sum_{j = 1}^N a_{ij}\left(\frac{r^\mathrm{new}_{ij}}{1 + \beta r^\mathrm{new}_{ij}} - \frac{r^\mathrm{old}_{ij}}{1 + \beta r^\mathrm{old}_{ij}}\right).
\end{eqnarray}

Exponentiating both sides reveals the final optimized ratio 

\cfbox{4.6cm-2pt}{
\begin{equation}
 \frac{J^\mathrm{new}}{J^\mathrm{old}} = \exp \left[\sum_{j = 1}^N a_{ij}\left(\frac{r^\mathrm{new}_{ij}}{1 + \beta r^\mathrm{new}_{ij}} - \frac{r^\mathrm{old}_{ij}}{1 + \beta r^\mathrm{old}_{ij}}\right)\right].
\end{equation}
}


\section{Optimizing the Padé Jastrow Derivative Ratios}


The shape of the Padé Jastrow factor is general in the sense that it is independent of the system at hand. Calculating closed form expressions for the derivatives is then a process which can be done once and for all. 

Closed form expressions are not only very efficient compared to a numerical evaluation, but also exact to machine precision. These facts renders closed form expressions of high interest to any Monte-Carlo implementation. 

\subsection{The Gradient}

Using the notation of Eq.~(\ref{eq:jastrowRatSTD}), the $x$-component of the Padé Jastrow gradient ratio for particle $i$ is

\begin{equation}
 \frac{1}{J}\frac{\partial J}{\partial x_i} = \frac{1}{\prod_{k < l}\exp g_{kl}}\frac{\partial }{\partial x_i}\prod_{k < l}\exp g_{kl}.
\end{equation}

Using the product rule, the above product will be transformed into a sum, where only the terms which has $k$ or $l$ equal to $i$ survive the differentiation. In addition, the terms independent of $i$ will cancel the corresponding terms in the denominator. Performing this calculation yields

\begin{eqnarray}
  \frac{1}{J}\frac{\partial J}{\partial x_i} &=& \sum_{k \ne i} \frac{1}{\exp g_{ik}}\frac{\partial}{\partial x_i} \exp g_{ik} \nonumber\\
  &=& \sum_{k \ne i} \frac{1}{\exp g_{ik}}\frac{\partial g_{ik}}{\partial x_i} \exp g_{ik}\nonumber \\
  &=& \sum_{k \ne i} \frac{\partial g_{ik}}{\partial x_i} \nonumber \\
  &=& \sum_{k \ne i} \frac{\partial g_{ik}}{\partial r_{ik}}\frac{\partial r_{ik}}{\partial x_i},
\end{eqnarray}

where

\begin{eqnarray}
 \frac{\partial g_{ik}}{\partial r_{ik}} &=& \frac{\partial }{\partial r_{ik}} \left(\frac{a_{ik}r_{ik}}{1 + \beta r_{ik}}\right) \nonumber\\
  &=& \frac{a_{ik}}{1 + \beta r_{ik}} - \frac{a_{ik}r_{ik}}{(1 + \beta r_{ik})^2}\beta \nonumber \\
  &=& \frac{a_{ik}(1 + \beta r_{ik}) - a_{ik}\beta r_{ik}}{(1 + \beta r_{ik})^2}  \nonumber \\
  &=& \frac{a_{ik}}{(1 + \beta r_{ik})^2}, \label{eq:jastrowDgikDrik}
\end{eqnarray}

and

\begin{eqnarray}
 \frac{\partial r_{ik}}{\partial x_i} &=& \frac{\partial }{\partial x_i} \sqrt{(x_i - x_k)^2 + (y_i - y_k)^2 + (z_i - z_k)^2} \nonumber \\
  &=& \frac{1}{2} 2(x_i - x_k) / \sqrt{(x_i - x_k)^2 + (y_i - y_k)^2 + (z_i - z_k)^2} \nonumber \\
  &=& \frac{x_i - x_k}{r_{ik}}.
\end{eqnarray}

Combining these expressions yields

\begin{equation}
\label{eq:jastrowDerivX}
 \frac{1}{J}\frac{\partial J}{\partial x_i} = \sum_{k \ne i} \frac{a_{ik}}{r_{ik}}\frac{x_i - x_k}{(1 + \beta r_{ik})^2}.
\end{equation}

Changing the Cartesian variable in the differentiation changes only the numerator of Eq.~(\ref{eq:jastrowDerivX}). In other words, generalizing to the full gradient is done by substituting the Cartesian difference with the position vector difference. The expression for the gradient becomes

\cfbox{4.8cm}{
\begin{equation}
\label{eq:jastrowGradFull}
 \frac{\nabla_i J}{J} = \sum_{k \ne i = 1}^N \frac{a_{ik}}{r_{ik}}\frac{\mathbf{r}_i - \mathbf{r}_k}{(1 + \beta r_{ik})^2}.
\end{equation}
}


\subsection{The Laplacian}

The same strategy used to obtain the closed form expression for the gradient in the previous section can be applied to the Laplacian. The full calculation is done in ref. \cite{morten}. The expression becomes

\begin{equation}
\label{eq:jastrowLaplRaw}
 \frac{\nabla^2_i J}{J} = \left| \frac{\nabla_i J}{J}\right|^2 + \sum_{k \ne i = 1}^N \left(\frac{d-1}{r_{ik}}\frac{\partial g_{ik}}{\partial r_{ik}} + \frac{\partial^2 g_{ik}}{\partial r_{ik}^2}\right) 
\end{equation}

where $d$ is the number of dimensions arising due to the fact that the Laplacian, unlike the gradient, is a summation of contributions from all dimensions. A simple differentiation of Eq.~(\ref{eq:jastrowDgikDrik}) with respect to $r_{ik}$ yields

\begin{equation}
\label{eq:jastrowD2gikDrik2}
 \frac{\partial^2 g_{ik}}{\partial r_{ik}^2} = -\frac{2a_{ik}\beta}{(1 + \beta r_{ik})^3}
\end{equation}

Inserting Eq.~(\ref{eq:jastrowDgikDrik}) and Eq.~(\ref{eq:jastrowD2gikDrik2}) into Eq.~(\ref{eq:jastrowLaplRaw}) further reveals

\begin{eqnarray}
 \frac{\nabla^2_i J}{J} &=& \left| \frac{\nabla_i J}{J}\right|^2 + \sum_{k \ne i = 1}^N \left(\frac{d-1}{r_{ik}}\frac{a_{ik}}{(1 + \beta r_{ik})^2} - \frac{2a_{ik}\beta}{(1 + \beta r_{ik})^3}\right) \nonumber\\
  &=& \left| \frac{\nabla_i J}{J}\right|^2 + \sum_{k \ne i = 1}^N a_{ik}\frac{(d-1)(1 + \beta r_{ik}) - 2\beta r_{ik}}{r_{ik}(1 + \beta r_{ik})^3}, \nonumber
\end{eqnarray}

which when cleaned up results in

\begin{equation}
 \frac{\nabla^2_i J}{J} = \left| \frac{\nabla_i J}{J}\right|^2 + \sum_{k \ne i} a_{ik}\frac{(d-3)(\beta r_{ik} + 1) + 2}{r_{ik}(1 + \beta r_{ik})^3}.
\end{equation}

The local energy calculation needs the sum of the Laplacians for all particles (see Eq.~(\ref{eq:localEsum})). In other words, the quantity of interest becomes

\begin{equation}
 \sum_i \frac{\nabla^2_i J}{J} = \sum_i \left[\left|\frac{\nabla_i J}{J}\right|^2 + \sum_{k \ne i}^N a_{ik}\frac{(d-3)(\beta r_{ik} + 1) + 2}{r_{ik}(1 + \beta r_{ik})^3}\right].
\end{equation}

Due to the symmetry of $r_{ik}$, the second term count equal values twice. Further optimization can thus be achieved by calculation only terms where $k>i$, and multiply the sum by two. Bringing it all together yields

\cfbox{5cm}{
\begin{equation}
\label{eq:jastrowLaplFin}
 \sum_i \frac{\nabla^2_i J}{J} = \sum_i \left|\frac{\nabla_i J}{J}\right|^2 + 2\sum_{k > i} a_{ik}\frac{(d-3)(\beta r_{ik} + 1) + 2}{r_{ik}(1 + \beta r_{ik})^3}
\end{equation}
}


\section{Tabulating Recalculated Data}

The expressions introduced so far in this chapter contain terms which are identical. Explicitly calculating these terms every time they are countered in the code would waste a lot of CPU time in contrast to the optimal scenario where they are calculated only once.

Tabulating data is crucial in order for the code to remain efficient for an increased number of particles. 

\subsection{The relative distance matrix}
\label{sec:relDist}

In the discussions regarding the optimization of the Jastrow ratio in section \ref{sec:optPagejastRat}, it became clear that moving one particle only changed $N$ of the relative distances. Tabulating these distances in a matrix $\mathbf{r}_\mathrm{rel}$, which then can be updated when a particle is moved, will ensure that the relative distances are calculated once and for all. The matrix is set up in the following manner

\begin{equation}
\mathbf{r}_\mathrm{rel} = \mathbf{r}_\mathrm{rel}^T = \left( \begin{array}{ccccc}
0 & r_{12} & r_{13} & \cdots & r_{1N} \\
r_{21} & 0 & r_{23} & \cdots & r_{2N}  \\
 &  & \ddots & \ddots & \vdots \\
 & \cdots &  & 0 & r_{(N-1)N} \\
 &  &  &  & 0\end{array} \right),
\end{equation}

where the first equality expresses that the matrix is symmetric, that is, equal to it's own transpose. The following code presents the updating of the matrix when a particle is moved

\vspace{0.25cm}
\begin{lstlisting}[caption={The code used to update the relative distance matrix of a walker when a particle is moved. Pay extra attention to line 8 where the symmetry of the matrix is exploited to further decrease the number of calls to the relative distance function in line 9.}]
void Sampling::update_pos(const Walker* walker_pre, Walker* walker_post, int particle) const {

    ...

    //Updating the part of the r_rel matrix which is changed by moving the [particle]
    for (int j = 0; j < n_p; j++) {
        if (j != particle) {
            walker_post->r_rel(particle, j) = walker_post->r_rel(j, particle)
                    = walker_post->calc_r_rel(particle, j);
        }
    }
    
    ...

}
\end{lstlisting}


Functions such as \verb+Coulomb::get_potential_energy+ and all of the Jastrow functions can then simply access these matrix elements without having to perform any explicit calculations.

A similar updating algorithm has been implemented for the squared distance vector and the absolute value of the distance vector. 

\subsection{The Slater related matrices}
\label{sec:storeSlater}

Apart from the inverse, whose optimization was covered in Section \ref{sec:optInv}, calculating the single particle wave functions and its gradients are the most expensive operations in the QMC algorithm.

Storing these function values in matrices representing the Slater matrices $\mathbf{S}^\uparrow$ and $\mathbf{S}^\downarrow$ introduced in Section \ref{sec:optSingleSlater}, ensures that these values never gets recalculated. The following expression describes how the Slater matrix is represented in the code 

\begin{equation}
\label{eq:slaterConcat}
 \mathbf{S}  \equiv \mathrm{join}\left(\mathbf{S}^\uparrow,\,\,\mathbf{S}^\downarrow\right) = \left[ \begin{array}{cccc}
\phi_1(\mbf{r}_0)     & \phi_1(\mbf{r}_1)     & \cdots & \phi_1(\mbf{r}_N)       \\
\phi_2(\mbf{r}_0)     & \phi_2(\mbf{r}_1)     & \cdots & \phi_2(\mbf{r}_N)       \\
\vdots          & \vdots          &        & \vdots            \\
\phi_{N/2}(\mbf{r}_0) & \phi_{N/2}(\mbf{r}_1) & \cdots & \phi_{N/2}(\mbf{r}_N)   \end{array} \right],
\end{equation}

where $\mathrm{join}(\mathbf{A}, \mathbf{B})$ denotes joining the columns of the matrices $A$ and $B$, i.e. \textit{concatenate} the matrices\cite{linAlg}. Similarly for the gradient terms, the following matrix is responsible for storing the gradient of all the elements in Eq.~(\ref{eq:slaterConcat})

\begin{equation}
\label{eq:slaterDellConcat}
 \mathbf{dS} \equiv \left[ \begin{array}{cccc}
\nabla \phi_1(\mbf{r}_0)     & \nabla \phi_1(\mbf{r}_1)     & \cdots & \nabla \phi_1(\mbf{r}_N)       \\
\nabla \phi_2(\mbf{r}_0)     & \nabla \phi_2(\mbf{r}_1)     & \cdots & \nabla \phi_2(\mbf{r}_N)       \\
\vdots                 & \vdots                 &        & \vdots                   \\
\nabla \phi_{N/2}(\mbf{r}_0) & \nabla \phi_{N/2}(\mbf{r}_1) & \cdots & \nabla \phi_{N/2}(\mbf{r}_N)   \end{array} \right].
\end{equation}

The inverse Slater matrices are implemented this way as well:

\begin{equation}
 \Sinv \equiv \mathrm{join}\left((\mathbf{S}^\uparrow)^{-1}\,\,,(\mathbf{S}^\downarrow)^{-1}\right).
\end{equation}

In the code, these matrices are stored as \verb+walker.phi+, \verb+walker.del_phi+ and \verb+walker.inv+. When one particle is moved, only a single row in the two first matrices needs to be recalculated, and only half of the concatenated inverse needs to be updated.

This concatenation of matrices ensures that no conditional tests are needed in order to access the correct matrix for a given particle. 

\subsection{The Padé Jastrow gradient}
\label{sec:optJastGrad}

Just as for the Jastrow Laplacian, there are symmetries in the expression for the gradient in Eq.~(\ref{eq:jastrowGradFull}), which implies an optimized way of calculating it. However, unlike the Laplacian, the gradient is split into components, which makes the exploitation of symmetries a little less straight-forward. 

Defining

\begin{equation}
 \mathrm{d}\mbf{J}_{ik}  \equiv \frac{a_{ik}}{r_{ik}}\frac{\mathbf{r}_i -  \mathbf{r}_k}{(1 + \beta r_{ik})^2},
\end{equation}

it is apparent that 

\begin{equation}
 \mathrm{d}\mbf{J}_{ik} = -\mathrm{d}\mbf{J}_{ki}.
\end{equation}

Since the gradient can be written in terms of this quantity, that is,

\begin{equation}
 \frac{\nabla_i J}{J} = \sum_{k \ne i = 1}^N \mathrm{d}\mbf{J}_{ik},
\end{equation}

the code can be optimized by exploiting this antisymmetry. Consider the following matrix

\begin{equation}
\label{eq:jastrowDJ}
 \mathbf{\mathrm{d}J}\equiv \left( \begin{array}{ccccc}
0 & \mathrm{d}\mbf{J}_{12} & \mathrm{d}\mbf{J}_{13} & \cdots & \mathrm{d}\mbf{J}_{1N} \\
 -\mathrm{d}\mbf{J}_{12} & 0 & \mathrm{d}\mbf{J}_{23} & \cdots & \mathrm{d}\mbf{J}_{2N}  \\
 &  & \ddots & \ddots & \vdots \\
 & (-) &  & 0 & \mathrm{d}\mbf{J}_{(N-1)N} \\
 &  &  &  & 0\end{array} \right).
\end{equation}

In the same way that the relative distance matrix was equal to its transpose, the matrix above is equal to the negative of its transpose. In other words:

\begin{equation}
 \mathrm{d}\mathbf{J} = -\mathrm{d}\mathbf{J}^T.
\end{equation}

Additionally, just as for the relative distances, moving a single particle only changes one row and one column in the matrix. This implies that a similar updating algorithm as the one discussed in section \ref{sec:relDist} can be implemented. This is demonstrated in the following code snippet

\vspace{0.25cm}
\begin{lstlisting}[caption={The updating algorithm for the three-dimensional matrix used in the Padé Jastrow gradient. Pay close attention to line 11, where the symmetry property is exploited by setting the transposed term equal to the negative of an already calculated term.}]
void Pade_Jastrow::get_dJ_matrix(Walker* walker, int moved_particle) const {
    
    int i = moved_particle;
    for (int j = 0; j < n_p; j++) {
	if (j == i) continue;
	
        b_ij = 1.0 + beta * walker->r_rel(i, j);
        factor = a(i, j) / (walker->r_rel(i, j) * b_ij * b_ij);
        for (int k = 0; k < dim; k++) {
            walker->dJ(i, j, k) = (walker->r(i, k) - walker->r(j, k)) * factor;
            walker->dJ(j, i, k) = -walker->dJ(i, j, k);
        }
    }
}
\end{lstlisting}


Calculating the new gradient is now only a matter of summing the rows of the matrix in Eq.~(\ref{eq:jastrowDJ}):

\begin{equation}
\frac{\nabla_i J^\mathrm{old}}{J^\mathrm{old}} = \sum_{k \ne i = 1}^N \mathrm{d}\mbf{J}^\mathrm{old}_{ik}.\label{eq:jastrowGraDJold}
\end{equation}

Further optimization can be achieved by realizing that the function which calculates the new gradient also has access to the gradient of the previous iteration

\begin{equation}
 \frac{\nabla_i J^\mathrm{new}}{J^\mathrm{new}} = \sum_{k \ne i = 1}^N \mathrm{d}\mbf{J}^\mathrm{new}_{ik}. \label{eq:jastrowGradDJnew}
\end{equation}

As mentioned previously, moving a particle, $p$, only a single row and column in $\mathrm{d} J$. For all other particles $i\ne p$, only a single term from the new matrix is required to update the old gradient

\begin{equation}
 \frac{\nabla_{i\ne p} J^\mathrm{new}}{J^\mathrm{new}} = \sum_{k = 1}^N \mathrm{d}\mbf{J}^\mathrm{new}_{ik} = \left[\sum_{k \ne p} \mathrm{d}\mbf{J}^\mathrm{old}_{ik}\right] + \mathrm{d}\mbf{J}^\mathrm{new}_{ip}.
\end{equation}

Adding and subtracting the term missing from the sum to make it equal to the old gradient from Eq.~(\ref{eq:jastrowGraDJold}) gives

\begin{eqnarray}
 \frac{\nabla_{i\ne p} J^\mathrm{new}}{J^\mathrm{new}} &=& \left[\sum_{k \ne p} \mathrm{d}\mbf{J}^\mathrm{old}_{ik} + \mathrm{d}\mbf{J}^\mathrm{old}_{ip}\right] - \mathrm{d}\mbf{J}^\mathrm{old}_{ip} + \mathrm{d}\mbf{J}^\mathrm{new}_{ip} \\
 &=& \frac{\nabla_i J^\mathrm{old}}{J^\mathrm{old}} - \mathrm{d}\mbf{J}^\mathrm{old}_{ip} + \mathrm{d}\mbf{J}^\mathrm{new}_{ip}, \label{eq:jastrowGradOptDJ}
\end{eqnarray}

effectively reducing the calculation to three flops. For the case where $i=p$, the entire sum must be calculated as in Eq.~(\ref{eq:jastrowGradDJnew}). This process is demonstrated in the following code

\vspace{0.25cm}
\begin{lstlisting}[caption={The implementation of the Padé Jastrow gradient using the matrix from Eq.~(\ref{eq:jastrowDJ}). Lines 18-25 describe the case for gradients not equal to the moved particle, i.e. Eq.~(\ref{eq:jastrowGradOptDJ}). Lines 5-18 describe the case for the gradient of the moved particle, where the full sum is calculated as Eq.~(\ref{eq:jastrowGradDJnew}).}]
void Pade_Jastrow::get_grad(const Walker* walker_pre, Walker* walker_post, int p) const {
    double sum;

    for (int i = 0; i < n_p; i++) {
        if (i == p) {
       
            //for i == p the entire sum needs to be calculated
            for (int k = 0; k < dim; k++) {

                sum = 0;
                for (int j = 0; j < n_p; j++) {
                    sum += walker_post->dJ(p, j, k);
                }

                walker_post->jast_grad(p, k) = sum;
            }
        
        } else {
       
            //for i != p only one term differ from the old and the new matrix
            for (int k = 0; k < dim; k++) {
                walker_post->jast_grad(i, k) = walker_pre->jast_grad(i, k)
                        + walker_post->dJ(i, p, k) - walker_pre->dJ(i, p, k);
            }
        }
    }
}
\end{lstlisting}

The dimensions of $\mathrm{d}\mathbf{J}$ are $N\times N\times d$, where $N$ is the number of particles and $d$ is the dimension. This implies that the optimizations in the Jastrow gradient discussed in this section scales very well with $N$. See Section \ref{sec:optRes} for a demonstration of the speedup in a $N=30$, $d=2$ case. 


\subsection{The single-particle Wave Functions}
\label{sec:optSPWFqnumIndie}

For systems of many particles, the function call \verb+Orbitals::phi(walker, i, qnum)+ needs to figure out which expression is related to which quantum number. The brute force implementation is to simply perform an if test on the quantum number, and use this to return the corresponding expression. This is demonstrated in the following code

\clearpage
\vspace{0.25cm}
\begin{lstlisting}
double AlphaHarmonicOscillator::phi(const Walker* walker, int particle, int q_num) {
    
    //Ground state of the harmonic oscillator
    if (q_num == 0){
      return exp(-0.5*w*walker->get_r_i2(i));
    }
    
    ...
    
}
\end{lstlisting}

For a low number of particles this is quite efficient, however, this is not the case for a large number of particles. 

A more efficient implementation is to represent the single particle wave functions as \verb+BasisFunctions+ objects. These objects holds only one pure virtual member function \verb+BasisFunctions::eval()+ which takes on input the particle number $i$ and the walker and returns the evaluated single particle wave function. The object itself is defined by a quantum number $q$. 

The following is an example of a two-dimensional harmonic oscillator single particle wave function for quantum number $q=1$ (see Appendix \ref{appendix:SymPyHO} for $\phi_1$)

\vspace{0.25cm}
%[caption={The implementation of a two-dimensional harmonic oscillator single particle wave function representing quantum number 1.}]
\begin{lstlisting}
double HarmonicOscillator_1::eval(const Walker* walker, int i) {

    y = walker->r(i, 1);
    
    //y*exp(-k^2*r^2/2)
    
    H = y;
    return H*exp(-0.5*w*walker->get_r_i2(i));
    
}
\end{lstlisting}

These objects representing single particle wave functions can be loaded into an array in such a way that element $q$ corresponds to the \verb+BasisFunctions+ object representing this quantum number, e.g.~\\\verb+basis_functions[1] = new HarmonicOscillator_1()+. The new \verb+Orbitals::phi+ implementation then simply becomes a call to an array. This is demonstrated in the following code

\vspace{0.25cm}
\begin{lstlisting}[caption={The implementation of the single particle basis used in the code. It is simply a call to an array holding all the different single particle wave functions. The quantum number is used as an index, and the corresponding evaluation function is called with the supplied walker for the given particle.}]
double Orbitals::phi(const Walker* walker, int particle, int q_num) {
    return basis_functions[q_num]->eval(walker, particle);
}
\end{lstlisting}

All discussed optimizations thus so far are general in the sense that they are independent of the explicit shape of the single particle wave functions. There should, however, be room for optimizations in the basis functions themselves, as long as these are applied locally within each class where the explicit shape of the orbitals are absolute.

As discussed previously, only a single column in the Slater related matrices from Section \ref{sec:storeSlater} needs to be updated when a particle is moved. This implies that the terms which are independent of the quantum numbers can be calculated once for every particle instead of once for every quantum number. 

These terms often come in the shape of exponentials factors, which are conserved in derivatives, implying that these terms appear in both the gradients and the Laplacians as well as in the single particle wave functions themselves.

Looking at the harmonic oscillator - and the hydrogen-like wave functions listed in Appendix \ref{appendix:SymPyHO} - \ref{appendix:SymPyHydro}, the exponential factors are indeed independent of the quantum numbers in all of the terms. Referring to the quantum number independent terms as $\overline{Q}_i$, the expressions are 

\begin{eqnarray}
\overline{Q}_i^\mathrm{H.O.} &=& e^{-\frac{1}{2}\alpha\omega r_i^2} \\
 \overline{Q}_i^\mathrm{Hyd.} &=& e^{-\frac{1}{n}\alpha Z r_i}
\end{eqnarray}

The hydrogen eigenstates has a dependence on the principal quantum number $n$ in the exponential, nevertheless several expressions share this exponential factor. Calculating for example the $n=2$ exponentials beforehand saves $19$ exponential calls per particle every cycle, resulting in a dramatic speedup.

Implementing these quantum number independent factors are very simple; upon moving a particle, the virtual function \verb+Orbitals::set_qnum_indie_terms+ is called, which updates the value of the \verb+exp_factor+  pointer shared by all the loaded \verb+BasisFunctions+ objects and the \verb+Orbitals+ class. The following code snippet presents implementations of the function in case of the harmonic oscillator - and the hydrogen-like basis


\vspace{0.5cm}
\begin{lstlisting}[caption={Implementation of the function handling the calculation of the quantum number independent terms. Lines 1-5 describe the harmonic oscillator case, where the exponential factor $\exp(-\alpha\omega r_i^2)$ is the independent factor. Lines 7-21 describe the hydrogen-like case, where the calculated exponential factor $\exp(-\alpha Z |r_i|/n)$ has a dependence on the principal quantum number $n$. One factor is thus calculated per $n$, however, if no particles occupy states with a given $n$, the corresponding factor is not calculated (see lines 17-19). }]
void AlphaHarmonicOscillator::set_qnum_indie_terms(const Walker * walker, int i) {
   
   //k2 = alpha*omega 
   *exp_factor = exp(-0.5 * (*k2) * walker->get_r_i2(i));
}

void hydrogenicOrbitals::set_qnum_indie_terms(Walker* walker, int i) {

    //waler::calc_r_i() calculates |r_i| such that walker::get_r_i() can be used
    walker->calc_r_i(i);

    //k = alpha*Z 
    double kr = -(*k) * walker->get_r_i(i);
    
    //Calculates only the exponentials needed based on the number of particles
    *exp_factor_n1 = exp(kr);
    if (n_p > 2) *exp_factor_n2 = exp(kr / 2);
    if (n_p > 10) *exp_factor_n3 = exp(kr / 3);
    if (n_p > 28) *exp_factor_n4 = exp(kr / 4);

}
\end{lstlisting}

The \verb+BasisFunctions+ objects share the pointer to the correct exponential factor with the orbital class. These exponential factors can then simply be accessed instead of being recalculated numerous times. The following code demonstrates accessing the exponential factor calculated in the previous code example

\begin{lstlisting}[caption={The implementation of a single particle wave function. The pointer to the previously calculated exponential factor is simply accessed in line 8.}]
double HarmonicOscillator_1::eval(const Walker* walker, int i) {

    y = walker->r(i, 1);
    
    //y*exp(-k^2*r^2/2)
    
    H = y;
    return H*(*exp_factor);
    
}
\end{lstlisting}

For two-dimensional quantum dots, $112$ \verb+BasisFunctions+ objects are needed for a $56$-particle simulation. Applying the currently discussed optimization reduces the number of exponential calls needed to calculate every wave function from $112$ to $1$, which for an average DMC calculation results in $~6\cdot 10^{11}$ saved exponential calls. Generally, the number of exponential calls are reduced by a factor $\frac{N}{2}(N + 1)$.


\section{CPU Cache Optimization}
\label{sec:CPUcache}

The \textit{CPU cache} is a limited amount of memory directly connected to the CPU, designed to reduce the average time to access memory. Simply speaking, standard memory is slower than the CPU cache, as bits have to travel through the motherboard before it can be fed to the CPU (a so called \textit{bus}). 

Which values are held in the CPU cache is controlled by the compiler, however, if programmed poorly, the compiler will not be able to handle the cache storage optimally. Optimization tools such as \verb+O3+ exist in order to work around this, however, keeping the cache in mind from the beginning of the coding process may result in a much faster code. In the case of the QMC code, the most optimal use of the cache would be to have all the active walkers in the cache at all times. 

The memory is sent to the cache as arrays, which means that storing walker data sequentially in memory is the way to go in order to make take full use of the processor cache. If objects are declared as pointers, which is the case for matrices of general sizes, the memory layout is uncontrollable. This fact renders a QMC solver for a general number of particles hard to optimize with respect to the cache.







